\RequirePackage[switch, columnwise, running, mathlines, displaymath,
mathlines]{lineno}
\RequirePackage{docswitch}
% \flag is set by the user, through the makefile:
%    make note
%    make apj
% etc.
\setjournal{\flag}

\documentclass[\docopts]{\docclass}

\newcommand*\patchAmsMathEnvironmentForLineno[1]{%
  \expandafter\let\csname old#1\expandafter\endcsname\csname #1\endcsname
  \expandafter\let\csname oldend#1\expandafter\endcsname\csname end#1\endcsname
  \renewenvironment{#1}%
     {\linenomath\csname old#1\endcsname}%
     {\csname oldend#1\endcsname\endlinenomath}}%
\newcommand*\patchBothAmsMathEnvironmentsForLineno[1]{%
  \patchAmsMathEnvironmentForLineno{#1}%
  \patchAmsMathEnvironmentForLineno{#1*}}%
\AtBeginDocument{%
\patchBothAmsMathEnvironmentsForLineno{equation}%
\patchBothAmsMathEnvironmentsForLineno{align}%
\patchBothAmsMathEnvironmentsForLineno{flalign}%
\patchBothAmsMathEnvironmentsForLineno{alignat}%
\patchBothAmsMathEnvironmentsForLineno{gather}%
\patchBothAmsMathEnvironmentsForLineno{multline}%
}

% You could also define the document class directly
%\documentclass[]{emulateapj}

% Custom commands from LSST DESC, see texmf/styles/lsstdesc_macros.sty
\usepackage{lsstdesc_macros}
% \usepackage{natbib}
% \usepackage{subcaption}
\usepackage[colorinlistoftodos]{todonotes}
% \usepackage[colorlinks=true, allcolors=blue]{hyperref}
\usepackage{multirow}
\usepackage{xspace}
\usepackage{enumitem}
\usepackage{graphicx}
\usepackage{wasysym}
\usepackage{amsmath}
\usepackage{amsfonts}
\graphicspath{{./}{./figures/}}
\bibliographystyle{mnras}%mn2e

% Add your own macros here:

\newcommand{\lsst}{\textsc{LSST}}
\newcommand{\lsstdesc}{\lsst\textsc{-DESC}}
\newcommand{\pz}{photo-$z$}
\newcommand{\pzpdf}{\pz\ PDF}% can easily change this to specify posterior throughout
\newcommand{\qp}{\texttt{qp}}

\newcommand{\trainz}{\textsc{trainZ}}

\newcommand{\textul}{\underline}
\newcommand*\mathinhead[2]{\texorpdfstring{$\boldsymbol{#1}$}{#2}}

\newcommand{\red}[1]{\textcolor{red}{#1}}
% Claim a color for your comments
\newcommand{\aim}[1]{\textcolor{green}{#1}}%Alex Malz comments in green
\newcommand{\blue}[1]{\textcolor{blue}{#1}}%unknown person comments in blue
\definecolor{scc}{rgb}{0.0, 0.26, 0.15}
\newcommand{\scc}[1]{\textcolor{scc}{#1}}%Stefano Cavuoti Comments in dark green
\definecolor{newpink}{rgb}{0.858, 0.188, 0.478}
\newcommand{\erfan}[1]{\textcolor{newpink}{#1}} % Erfan Nourbakhsh comments in pink
\newcommand{\jan}[1]{\textcolor{orange}{#1}}%Alex Malz comments in green

\def\X{{\mathbf{X}}}
\def\x{{\mathbf{x}}}
\def\E{{\mathbb{E}}}

% ======================================================================

\begin{document}
\linenumbers

\title{Implicit assumptions and their impact on photometric redshift PDF performance for LSST}
% \title{Photometric redshift uncertainties for LSST \romannumeral{1}: A unified methodology for assessing photo-$z$ PDF performance}

\maketitlepre

\begin{abstract}

In order to maximize scientific returns of current and upcoming galaxy surveys, the photometric redshift (photo-$z$) posterior distributions produced by redshift estimation codes must be accurate probability distribution functions (PDFs).  However, the posteriors resulting from a number of current techniques are not, in general, consistent with each other, affected by implicit assumptions made by each code, and an optimal method for obtaining an accurate PDF estimate remains unclear.   We present the results of an initial study by the Large Synoptic Survey Telescope Dark Energy Science Collaboration (\textsc{LSST-DESC}) testing twelve photo-$z$ algorithms using complete and representative training data and evaluate multiple metrics to test how accurately the posteriors represent probability distributions.  We observe several trends, including systematic biases and an overall over/under-prediction in the broadness of the PDFs in many of the codes which may be symptomatic of implementation problems or problems in underlying algorithm design.  A careful accounting of all photo-$z$ systematics will be necessary for the codes employed in upcoming analyses in order to achieve unbiased cosmological measurements.
  
%  Photometric redshift (photo-$z$) probability distribution functions (PDFs) are a key data product of nearly all upcoming galaxy imaging surveys.  However, the photo-$z$ PDFs resulting from different techniques are not in general consistent with one another, and an optimal method for obtaining an accurate PDF remains unclear.  We present the results of an initial study of the Large Synoptic Survey Telescope Dark Energy Science Collaboration (\textsc{LSST-DESC}), the first in a planned series of papers testing multiple photo-$z$ codes on simulations of upcoming LSST galaxy photometry catalogues.  This initial test evaluates photo-$z$ algorithms in the presence of representative training data and in the absence of several common sources of systematic errors that affect the procedures by which photo-$z$ PDFs are derived.
%%  The photo-$z$ PDFs are evaluated using multiple metrics including the Kolmogorov-Smirnoff statistic, Cramer-von Mises statistic, Anderson-Darling statistic, Kullback-Leibler divergence, $N(z)$ moments, quantile-quantile plots and probability integral transform.
%  The photo-$z$ PDFs are evaluated using multiple metrics and we observe several trends, including an overall over/under-prediction in the broadness of the PDFs for several of the codes.  A careful accounting of all systematics discovered will be necessary for the codes employed in upcoming analyses in order to achieve unbiased cosmological measurements.
 %% Photometric redshift (photo-$z$) probability distribution functions (PDFs) more completely capture the complex nature of photo-$z$ uncertainties and are thus a key data product of nearly all upcoming galaxy imaging surveys, including the Large Synoptic Survey Telescope (LSST).
  % Photo-$z$ PDFs produced by different algorithms do not in general agree with one another, and no one technique has yet been identified as optimal, leading to a proliferation of photo-$z$ PDF estimation codes.
  % We present a unified comparison of twelve such codes using simulated LSST photometry, the first in a series of papers on photo-$z$ PDF performance for the LSST Dark Energy Science Collaboration (DESC).
  % In this initial study, we isolate the effects of the estimation procedure by running each code using perfectly representative training data (for machine learning methods) or the true SED template library (for template-fitting methods) and confirm the mutual inconsistency of the resulting photo-$z$ PDF estimates.
  % We apply a comprehensive suite of metrics to an the population of photo-$z$ PDFs as well as point estimates derived thereof and an accepted estimator of the redshift distribution $N(z)$.
  % % The photo-$z$ PDFs are evaluated using multiple metrics including the Kolmogorov-Smirnoff statistic, Cramer-von Mises statistic, Anderson-Darling statistic, Kullback-Leibler divergence, $N(z)$ moments, quantile-quantile plots, and probability integral transform.
  % We interpret the performance of the codes under each metric in terms of the assumptions underlying the overall approach of the methods and audit the sentivity of the metrics to the failure modes of different techniques, culminating in a discussion predicting the impact of introducing realistic systematics and the direction of attention to the choice of metric when identifying the most appropriate code for a given science use case.

\end{abstract}

% Keywords are ignored in the LSST DESC Note style:
\dockeys{galaxies: distances and redshifts -- galaxies: statistics -- methods: statistical}

\maketitlepost

% ----------------------------------------------------------------------
% 

\section{Introduction}
\label{sec:intro}

The current and next generations of large-scale galaxy surveys, including the Dark Energy Survey \citep[\textsc{DES},][]{Abbott:05}, the Kilo-Degree Survey \citep[\textsc{KiDS},][]{de_Jong:13}, Hyper Suprime-Cam Survey \citep[\textsc{HSC},][]{Aihara:2018a,Aihara:2018b}, Large Synoptic Survey Telescope \citep[\lsst,][]{Abell:09}, Euclid \citep{Laureijs:11}, and Wide-Field Infrared Survey Telescope \citep[\textsc{WFIRST},][]{Green:12}, present a paradigm shift from spectroscopic to photometric galaxy catalogs of substantially larger size at a cost of lacking complete spectroscopically confirmed redshifts.

Effective astrophysical inference using the catalogs resulting from these ongoing and upcoming missions, however, necessitates accurate and precise photometric redshift (\pz) estimation methodologies.
As an example, in order for \pz\ systematics to not dominate the statistical noise floor of \lsst's main cosmological sample of $\sim 10^{7}$ galaxies, the \lsst\ Science Requirements Document (SRD)\footnote{available at \url{https://docushare.lsstcorp.org/docushare/dsweb/Get/LPM-17}} specifies that individual galaxy \pz s must have root-mean-square error $\sigma_z < 0.02 (1+z)$, $3 \sigma$ ``catastrophic outlier'' rate below $10\%$, and bias below $0.003$.
Specific science cases may have their own requirements on \pz\ performance that exceed those of the survey as a whole.
In that vein, the \lsst\ Dark Energy Science Collaboration (\lsstdesc) developed a separate SRD \citep{Mandelbaum:2018} that conservatively forecasts the constraining power of five cosmological probes, leading to even more stringent requirements on \pz\ performance, including those defined in terms of tomographically binned subsamples populations rather than individual galaxies.

Though the standard has long been for each galaxy in a photometric catalog to have a \pz\ point estimate and Gaussian error bar, the nontrivial mapping between broad band fluxes and redshift renders this simplistic summary inadequate to quantify the uncertainty landscape by neglecting degenerate redshift solutions.
Far from a hypothetical situation, this degeneracy is a real consequence of the same deep imaging that enables larger galaxy catalog sizes.
The lower luminosity and higher redshift populations captured by deeper imaging introduce major physical systematics to \pz s, among them the Lyman break/Balmer break degeneracy, that did not affect shallower large area surveys like the Sloan Digital Sky Survey \citep[\textsc{SDSS},][]{York:00} and Two Micron All Sky Survey \citep[\textsc{2MASS},][]{Skrutskie:06}.

To fully characterize such physical degeneracies, photometric galaxy catalog data releases from \citep{Mandelbaum:2008} to \citep{de_Jong:17}, provided a more informative \pz\ data product, the \pz\ probability density function (PDF), that describes the relative probability as a function of a galaxy's redshift, conditioned on the observed photometry.
Early template-based methods such as \citet{Fernandezsoto:99} approximated the likelihood of photometry conditioned on redshift with the relative $\chi^{2}$ values of template spectra.
Not long after, Bayesian adaptations of template-based approaches such as \citet{Benitez:00} combined the estimated likelihoods with a prior to yield a posterior PDF of redshift conditioned on photometry.
While the first machine learning based algorithms focused on a point-estimate, \citet{Firth:03} estimated a \pzpdf\ using a neural net with 1000 realizations scattered within the photometric errors.

% MOVE TO WHERE \nz\ IS INTRODUCED AS A METRIC
% For cosmological measurements, certain science cases require redshift information on individual objects, e.~g.~identification of host galaxy redshift for supernova classification, or identifying potential cluster membership.
% Other science cases seem to need only ensemble redshift information; for instance many current cosmic shear techniques require only the overall redshift distribution $N(z)$ for tomographic redshift samples.
% However,  even such cases require individual object redshift estimates for portions of the analysis, for example in determining galaxy intrinsic alignments in weak lensing samples.
% In addition, recent data-driven techniques employing hierarchical Bayesian or Gaussian Process methods have emerged that calibrate redshift distributions using individual $p(z)$ estimates \citep[e.~g.~][]{Sanchez:2018}.
% Techniques for using \pzpdf s have lagged behind the development of codes to produce them, however, all existing approaches assume that the \pzpdf\ for each galaxy is an accurate PDF, with failures when the assumption is violated.
% Thus, even methods that seem to need only ensemble $N(z)$ may actually require accurate $p(z)$ in order to meet stringent survey requirements.
% Large photometric surveys such as LSST must develop algorithms that simultaneously meet the needs of all science cases.
% In order to meet these ambitious goals for photo-$z$ accuracy, every aspect of photo-$z$ estimation will have to be optimized: the algorithms employed, both template and machine-learning based (both in design and implementation); the spectroscopic data used as a training set for machine learning algorithms or to estimate template sets and train Bayesian priors; and probabilistic catalogue compression schemes that balance information retention against limited storage resources.

There are numerous techniques for deriving \pzpdf s, yet no one method has yet been established as clearly superior.
Quantitative comparisons of \pz\ methods have been made before.
The \Pz\ Accuracy And Testing \citep[\textsc{PHAT},][]{Hildebrandt:10} effort focused on \pz\ point estimates derived from many photometric bands.
\citet{Rau:2015} introduced a new method for improving \pzpdf s using an ordinal classification algorithm.
\textsc{DES} compared several codes for \pz\ point estimates and a subset with \pzpdf\ information \citep{Sanchez:14} and examined summary statistics of \pzpdf s for tomographically binned galaxy subsamples \citep{Bonnett:16}.

This paper is distinguished by its focus on the evaluation criteria for \pzpdf s and interpretation thereof, as part of the mission of the , as a key project of the Photometric Redshift working group of the \lsstdesc\ laid out in the Science Roadmap (SRM)\footnote{Available at: \url{http://lsst-desc.org/sites/default/files/DESC_SRM_V1_1.pdf}}.
In this initial study, we focus on evaluating the performance of \pzpdf\ codes and PDF-specific performance metrics in a controlled experiment with complete and representative prior information (template libraries and training sets).
% Specific implementation choices in each code will influence the resultant posterior distributions, for example choice of prior parameterization in template-based codes, the bandwidth size chosen for machine learning based codes, or even the output format chosen for storing the PDF.
% We have attempted to minimize the impact of many of these factors when comparing codes, for example by using the same template set for all template-based codes, and using a training set that is drawn from the same underlying population as the test sample, to create a controlled environment in which to compare the photo-$z$ PDFs derived from each method.
% We explore a number of performance metrics in this paper that test whether the posterior estimates are actual PDFs.
% Comparing the relative performance of the codes enables us to evaluate whether each code is using information in an optimal way, and may reveal enhancements in some codes and deficiencies in others, either in the fundamental algorithm, or in specific implementation.
% Identifying and fixing failure modes within codes may aid us in reaching the stringent photo-$z$ performance goals set out for LSST.
% We note that these initial tests are a necessary requirement for photo-$z$ codes that will be used in cosmological analyses; however, meeting these requirements is only the first stage in the process, and can be thought of as an initial test under near perfect conditions to test for problems before further complexities are added in future analyses.

The outline of the paper is as follows: in \S\,\ref{sec:sims} we present the simulated data set; in \S\,\ref{sec:pzcodes} we describe the current generation codes employed in the paper; in \S\,\ref{sec:metrics} we discuss the interpretation of photo-$z$ PDFs in terms of metrics of accuracy; in \S\,\ref{sec:results} we show our results and compare the performance of the codes; in \S\,\ref{sec:discussion} we offer our conclusions and discuss future extensions of this work.


\section{Data}
\label{sec:sims}

In order to test the current generation of \pzpdf\ codes, we employ an existing simulated galaxy catalogue, described in detail in Section~\ref{sec:buzzard}.
The experimental conditions shared among all codes are motivated by the \lsst\ SRD requirements and implemented for machine learning and template-based \pzpdf\ codes according to the procedures of Sections~\ref{sec:buzztraining} and \ref{sec:buzztemplates} respectively.

\subsection{The \textsc{Buzzard-v1.0} simulation}
\label{sec:buzzard}

Our mock catalogue is derived from the \textsc{Buzzard}-highres-v1.0 catalogue (DeRose et al., in prep).
\textsc{Buzzard} is built on the \texttt{Chinchilla-400} \citep{Mao:15} dark matter-only N-body simulation consisting of $2048^{3}$ particles in a $400$ Mpc h$^{-1}$ box.
The lightcone was constructed from smoothing and interpolation between a set of time snapshots.
Dark matter halos were identified using the \texttt{Rockstar} software package \citep{Behroozi:13} and then populated with galaxies with stellar masses and absolute $r$-band magnitudes in the \sdss\ system determined using a sub-halo abundance matching model constrained to match both projected two-point galaxy clustering statistics and an observed conditional stellar mass function \citep{Reddick:13}.

To assign a spectrum to each galaxy, the Adding Density Dependent Spectral Energy Distributions (SEDs) procedure \citep[\texttt{ADDSEDS,}][Wechsler et al., in prep,]{DeRose:19} was used.
\texttt{ADDSEDS} uses a sample of $\sim 5\times 10^{5}$ galaxies from the magnitude-limited \sdss\ Data Release 6 Value Added Galaxy Catalogue \citep[NYU-VAGC,][]{Blanton:05} to train an empirical relation between absolute $r$-band magnitude, local galaxy density, and SED.
Each \sdss\ spectrum is parameterized by five weights corresponding to a weighted sum of five basis SED components using the \texttt{k-correct} software package\footnote{\url{http://kcorrect.org}} \citep{Blanton:07}.

Correlations between SED and galaxy environment were included so as to preserve the colour-density relation of galaxy environments.
The distance to the spatially projected fifth-nearest neighbour was used as a proxy for local density in the \sdss\ training sample.
For each simulated galaxy, a galaxy with similar absolute $r$-band magnitude and local galaxy density was chosen from the training set, and that training galaxy's SED was assigned to the simulated galaxy.

\subsubsection{Caveats}
\label{sec:buzzlimitations}

By necessity, \textsc{Buzzard} does not contain all of the complicating factors present in real data, and here we discuss the most pertinent ways that these limitations affect our experiment.
\textsc{Buzzard} includes only galaxies, not stars nor AGN.
The catalogue-based construction excludes image-level effects, such as deblending errors, photometric measurement issues, contamination from sky background (Zodiacal light, scattered light, etc.), lensing magnification, and Galactic reddening.

The \textsc{Buzzard} SEDs are drawn from a set of $\sim 5 \times 10^{5}$ SEDs, which themselves are derived from a five-component linear combination fit to $\sim 5 \times 10^{5}$ \sdss\ galaxies; thus the sample contains only galaxies that resemble linear combinations of those for which \sdss\ obtained spectra, and there are necessarily duplicates.
The linear combination SEDs also restrict the properties of the galaxy population to linear combinations of the properties corresponding to five basis templates, precluding the modeling of non-linear features such as the full range of emission line fluxes relative to the continuum.
The only form of intrinsic dust reddening comes from what is already present in the five basis SEDs via the training set used to create the basis templates, and linear combinations thereof do not span the full range of realistic dust extinction observed in galaxy populations.

While these idealized conditions limit the realism of our mock data, they are irrelevant to the controlled experimental conditions of this study, if anything assuring that differentiation in the performance of the \pzpdf\ codes is due to the inferential techniques rather than nuances in the data.

\subsection{\lsst-like mock observations}
\label{sec:observations}

Given the SED, absolute $r$-band magnitude, and true redshift of each simulated galaxy, we computed apparent magnitudes in the six \lsst\ filter passbands, $ugrizy$.
We assigned magnitude errors in the six bands using the simple model of \citet{Ivezic:08}, assuming achievement of the full 10-year depth, with a modification of fiducial \lsst\ total numbers of 30-second visits for photometric error generation: we assume 60 visits in $u$-band, 80 visits in $g$-band, 180 visits in $r$-band, 180 visits in $i$-band, 160 visits in $z$-band, and 160 visits in $y$-band.

As a consequence of adding Gaussian-distributed photometric errors, 2.0 per cent of our galaxies exhibit a negative flux in one or more bands, the vast majority of which are in the $u$-band.
We deem such negative fluxes \textit{non-detections} and assign a placholder magnitude of 99.0 in the catalogue to indicate to the \pzpdf\ codes that such galaxies would be ``looked at but not seen'' in multi-band forced photometry.

The full dataset thus covers $400$ square degrees and contains $238$ million galaxies of redshift $0 < z \leq 8.7$ down to $r = 29$.
Systematic inconsistencies with galaxy colors at $z > 2$ were observed, so the catalogue was limited to $0 < z \leq 2.0$.
To obtain a catalogue matching the \lsst\ Gold Sample, we imposed a cut of $i < 25.3$, which gives a signal-to-noise ratio $\gtrsim 30$ for most galaxies.
In order for statistical errors to be subdominant to the systematic errors we aim to probe, we further reduced the sample size to $<10^{7}$ galaxies by isolating $\sim 16.8$ square degrees selected from five separate spatial regions of the simulation.
We refer to this final set of galaxies as DC1, for the first \lsstdesc\ Data Challenge.

\subsection{Shared prior information}
\label{sec:controlled}

For the purpose of performing a controlled experiment that compares \pzpdf\ codes on equal footing as a baseline for a future sensitivity analysis, we take care to provide each with optimal prior information.
Redshift estimation approaches built upon physical modeling and machine learning alike have a notion of prior information considered beyond the photometry of the data for which redshift is to be constrained: that information is derived from a template library for a model-based code and a training set for a data-driven code.
In this initial study, we seek to set a baseline for a later comparison of the performance of \pzpdf\ codes under incomplete and non-representative prior information that will propagate differently in the space of data-driven and model-based algorithms.
However, for the baseline case of perfect prior information, physical modeling and machine learning codes can indeed be put on truly equal footing.
We outline the equivalent ways of providing all codes perfect prior information below.

\subsubsection{Training and test set division}
\label{sec:buzztraining}

Following the findings of \citet{Bernstein:10}, \citet{Newman:2015}, and \citet{Masters:2015} that only $\sim\!10^{4}$ spectra are necessary to calibrate \pz s to Stage IV requirements, we aimed to set aside a randomly selected training set of $3-5\times 10^{4}$ galaxies, $\sim 10$ per cent of the full sample.
After all cuts described above, we designated the \textit{DC1 training set} of $44\,404$ galaxies for which observed photometry, true SEDs, and true redshifts would be provided to all codes and the blinded \textit{DC1 test set} of $399\,356$ galaxies for which photometry alone would be provided to all codes and \pzpdf s would be requested.
The exact form of \lsst\ photometric filter transmission curves were also considered public information that could be used by any code.

\subsubsection{Template library construction}
\label{sec:buzztemplates}

We aimed to provide template-fitting codes with complete yet manageable library of templates spanning the space of SEDs of the DC1 galaxies.
We constructed $K=100$ representative templates from the $\sim 5 \times 10^{5}$ SEDs of the \sdss\ DR6 NYU-VAGC by using the five-dimensional vectors of SED weight coefficients described above.
After regularizing the SED weight coefficients $\in [0, 1]$, we ran a simple K-means clustering algorithm on the five-dimensional space of regularized SED weight coefficients of the \sdss\ galaxy sample.
The resulting clusters were used to define Voronoi cells in the space of weight coefficients, with centre positions corresponding to weights for the \texttt{k-correct} SED components, yielding the 100 SEDs that comprise the \textit{DC1 template set} provided to all template-based codes.
We did not, however, exclude from consideration template-based codes that made modifications in their use of these templates due to architecture limitations (as opposed to knowledge of the experimental conditions that could artificially boost the code's apparent performance), with deviations noted in Section~\ref{sec:pzcodes}.


\section{Methods}
\label{sec:pzcodes}

Here we summarize the twelve \pzpdf\ codes compared in this study, summarized in Table~\ref{tab:list_of_codes}, which include both established and emerging approaches in template fitting and machine learning.
Though not exhaustive, this sample represents codes for which there was sufficient expertise within the \lsstdesc\ Photometric Redshifts Working Group; the authors welcome interest from those outside \lsstdesc\ to have their codes assessed in future investigations that build upon this one.

\begin{table*}  %%% DATA TABLE %%%
\caption{List of \pzpdf\ codes featured in this study} \label{tab:list_of_codes}\resizebox{\textwidth}{!}{
\begin{tabular}{lll}
\hline
\bf Published code & \bf Type & \bf Public source code \\
\hline
\lephare~\citep{Arnouts:99}	   & template fitting	& \url{http://www.cfht.hawaii.edu/~arnouts/lephare.html} \\
\bpz~\citep{Benitez:00} 		   & template fitting	& \url{http://www.stsci.edu/~dcoe/BPZ/} \\
\eazy~\citep{Brammer:08}		   & template fitting & \url{https://github.com/gbrammer/eazy-photoz} \\
\annz~\citep{Sadeh:16}		     & machine learning	& \url{https://github.com/IftachSadeh/ANNZ} \\
\flexzboost~\citep{Izbicki:17} & machine learning & \url{https://github.com/tpospisi/flexcode}; \url{https://github.com/rizbicki/FlexCoDE}\\
%\textsc{Frankenz} 	& ML 	& \citet{Speagle:frankenz}	& \url{https://github.com/joshspeagle/frankenz} \\
\gpz~\citep{Almosallam:15b}	   & machine learning	& \url{https://github.com/OxfordML/GPz} \\
\metaphor~\citep{Cavuoti:17}   & machine learning	& \url{http://dame.dsf.unina.it}\\
\cmnn~\citep{Graham:17}        & machine learning & N/A \\
\skynet~\citep{Graff:14}       & machine learning & \url{http://ccpforge.cse.rl.ac.uk/gf/project/skynet/} \\
\tpz~\citep{Carrasco_Kind:13}	 & machine learning	& \url{https://github.com/mgckind/MLZ} \\
\delight~\citep{Leistedt:17}   & hybrid           & \url{https://github.com/ixkael/Delight} \\
\hline
\trainz 	                             & machine learning	& See Section~\ref{sec:method:trainz} \\
\end{tabular}}
\end{table*}

We describe the algorithms and implementations of the model-based and data-driven codes in Sections~\ref{sec:templatecodes} and \ref{sec:trainingcodes} respectively, with a straw-person approach included in Section~\ref{sec:trainz}.
% For each approach, we note how it accounts for photometric uncertainties, how it interprets negative fluxes, what its native output format of \pzpdf s is, and what, if any, preprocessing, including validation, of the prior information it performs.
% In the following sections, we use a consistent notation defined in Table~\ref{tab:variables}.
%
% \begin{table}
% 	\label{tab:variables}
% 	\caption{Definitions of variables used in outlining \pzpdf\ codes}
% 	\begin{tabular}{ll}
% 		\hline
% 		\bf Symbol & \bf Definition \\
% 		\hline
% 		\multirow{ 2}{*}{$b = 1,\dots,B$	& \multirow{ 2}{*}{number $B$ of photometric filters $b$;} \\
% 														& \lsst's $ugrizy$ filters used here correspond to $B=6$ \\
% 		\multirow{ 2}{*}{$m_{b} = 1,\dots,B$	& \multirow{ 2}{*}{number $B$ of photometric filters $b$;} \\
% 																										& \lsst's $ugrizy$ filters used here correspond to $B=6$ \\
% 		$z$											& redshift \\
% 		$T$											& galaxy SED \\
%
% 	\end{tabular}
% \end{table}

\subsection{Template-based Approaches}
\label{sec:templatecodes}

We test three publicly available and commonly used template-based codes that share the standard physically motivated approach of calculating model fluxes for a set of template SEDs on a grid of redshift values and evaluating a \chisq\ merit function using the observed and model fluxes.

\subsubsection{LePhare}
\label{sec:lephare}
%(C\'ecile Roucelle, Eric Nuss, Johann Cohen-Tanugi)

\lephare \footnote{\url{http://www.cfht.hawaii.edu/~arnouts/lephare.html}}\citep[Photometric Analysis for Redshift Estimate,][]{Arnouts:99,Ilbert:06} matches observed colors with those predicted from a template set, which can be semi-empirical or entirely synthetic, directly according to the likelihood $\chi^{2}(z, T, A) \equiv \sum_{b}^{N_{\rm{filt}}} \left((F^{\rm{obs}}_{b} - A F^{\rm{mod}}_{b}(T, z)) / \sigma^{\rm{obs}}_{b}\right)^{2}$ of normalization factor $A$, template $T$, and redshift $z$.
In words, the likelihood is a sum of observed flux error $\sigma_{b}^{\rm{obs}}$-weighted squared differences between the observed flux $F^{\rm{obs}}_{b}$ and the normalized predicted flux $F^{\rm{mod}}_{b}(T, z)$ in $N_{\rm{filt}}$ photometric filters $b$, which is the \lsst\ $ugrizy$ filters in this case.
The reported \pzpdf\ is an arbitrary normalization of the likelihood evaluated on the output redshift grid.
%\textsc{LePhare} has been used to produce the COSMOS2015 photo-$z$ catalogue \citep{Laigle:16}.

Here we use \lephare-v 2.2 with the DC1 template set of Section~\ref{sec:buzztemplates}.

\subsubsection{BPZ}
\label{sec:BPZ}
%(Sam Schmidt)

\bpz \footnote{\url{http://www.stsci.edu/~dcoe/BPZ/}} \citep[Bayesian Photometric Redshift,][]{Benitez:00} determines the likelihood $p(C \vert z, T)$ of a galaxy's observed colours $C$ for a set of SED templates $T$ at redshifts $z$.
The \bpz\ likelihood is related to the \chisq\ likelihood by $p(C \vert z, T) \propto \exp[- \chi^{2} / 2]$.
Given a Bayesian prior $p(z, T \vert m_{0})$ over apparent magnitude $m_0$ and assuming that the SED templates are spanning and exclusive, \bpz\ constructs the redshift posterior $p(z \vert C, m_0)$ by marginalizing over all SED templates as in \citep[Eq.~3 from][]{Benitez:00}, corresponding to setting the parameter \texttt{PROBS\_LITE=TRUE} in the \bpz\ parameter file.
The \bpz\ prior is the product of an SED template proportion that varies with apparent magnitude $p(T \vert m_{0})$ and a prior $p(z \vert T, m_{0})$ over the expected redshift as a function of apparent magnitude and SED template.

Here we test \bpz-v 1.99.3 with the DC1 template set of Section~\ref{sec:buzztemplates}.
To keep the number of free parameters manageable, the DC1 template set is pre-sorted by the rest-frame $u-g$ colour and split into three broad classes of SED template, equivalent to the E, Sp and Im/SB types in .
The Bayesian prior term $p(T \vert m_{0})$ was derived directly from the DC1 training set, and the other term $p(z \vert T, m_{0})$ was chosen to be the best fit for the eleven free parameters of the functional form of \citet{Benitez:00}.
%For photo-$z$ point estimates we use the \texttt{Z\_B} output parameter.
Prior to running the code, the non-detection placeholder magnitude was replaced with an estimate of the one-$\sigma$ detection limit for the undetected band as a proxy for a value close to the estimated sky noise threshold.

\subsubsection{EAZY}
\label{sec:eazy}
%(Rongpu Zhou)

\eazy \footnote{\url{https://github.com/gbrammer/eazy-photoz}} \citep[Easy and Accurate Photometric Redshifts from Yale,][]{Brammer:08} extends the basic \chisq\ fit procedure that defines template-fitting approaches.
The algorithm models the observed photometry with a linear combination of template SEDs at each redshift.
The best-fit SED is found by simultaneously fitting one, two, or all of the templates via \chisq\ minimization, which is distinct from marginalizing across all templates.
The minimized \chisq\ likelihood at each redshift is then combined with an apparent magnitude prior to obtain the redshift posterior PDF.
We note that the utilization of the best-fit SED rather than a proper marginalization does not lead to the correct posterior distribution, an implementation issue that has now been identified and will be addressed by the developers in the future.
\eazy\ can account for uncertainty in the template set by adding in quadrature to the flux errors an empirically derived template error as a function of redshift.

The SED-independent apparent magnitude prior was derived empirically from the DC1 training set.
The \eazy\ architecture cannot accept a template set other than the same five basis templates employed by \texttt{k-correct} when constructing the DC1 catalogue.
However, \eazy\ does feature a flexible \texttt{all-templates} mode, which fits the photometric data with a linear combination of the five basis templates.
We set the template error to zero since the same templates were in fact used to produce the DC1 photometry.

\subsection{Training-based Approaches}
\label{sec:trainingcodes}

We compared nine data-driven \pz\ estimation approaches, eight of which are described in this section and one of which is discussed in Section~\ref{sec:trainz}.
Because the algorithms differ more from one another and the techniques are relative newcomers to the astronomical literature, we provide somewhat more detail about the implementations below.

% Some aspects of data treatment were left to the individual code runners, for example, whether and how to split the DC1 training set for validation.
% The codes considered treated
% Another key difference is the treatment of non-detections in one or more bands.
% Some codes choose to ignore a band, others replace the value with either an estimate for the detection limit, the mean of other values in the training set, or another default value.
% There are varying conventions among training-based codes for treatment of non-detections, and no one prescription dominates in the photo-$z$ literature.
% The specific choices for each code affect the results, and contribute to the implicit prior influencing their output.
% However, we remind the reader that only 2.0 per cent of our sample has non-detections, almost exclusively in the u-band, and thus should not dominate the code performance differences.

\subsubsection{ANNz2}
\label{sec:annz2}
%(John Soo)

\annz \footnote{\url{https://github.com/IftachSadeh/ANNZ}} \citep{Sadeh:16} employs several machine learning algorithms, including artificial neural networks (ANN), boosted decision tree, and k-nearest neighbour (KNN) regression.
In addition to accounting for errors on the input photometry, \annz\ uses the KNN-uncertainty estimate of \citet{Oyaizu:08} to quantify uncertainty in the choice of method over multiple runs.
Using the Toolkit for Multivariate Data Analysis with ROOT\footnote{\url{http://tmva.sourceforge.net/}}, it can return the results of running a single machine learning algorithm, a ``best'' choice of the results from simultaneously running multiple algorithms, or a combination of the results of multiple algorithms weighted by their method uncertainties averaged over multiple runs.
%\textsc{ANNz2} is capable of producing both photo-$z$ point estimates and redshift posterior probability distributions $p(z)$.
% It can also perform classifications, and supports reweighting between samples.
% \annz\ propagates the intrinsic uncertainty on the input parameters and the uncertainty in the machine learning method to the expected photo-$z$ solution, averaged over multiple runs weighted based on the performance of each run.

In this study, we used \annz-v.2.0.4 to output only the result of the ANN algorithm.
\Pzpdf s were produced by running an ensemble of 5 ANNs with a $6:12:12:1$ architecture corresponding to the 6 $ugrizy$ inputs, 2 hidden layers with 12 nodes each, and 1 output of redshift.
Each of the five ANNs was trained with different random seeds for the initialization of input parameters.
Additionally, all ANNs were trained on only a $i \leq 25.3$ subsample of the DC1 training set, and half of the training set was reserved for validation to prevent overfitting.
Undetected galaxies were excluded from the training set, and per-band non-detections in the test set were replaced with the mean magnitude in that band within the entire test set.

\subsubsection{Colour-Matched Nearest-Neighbours}
\label{sec:cmnn}
%(Melissa Graham)

The nearest-neighbours colour-matching photometric redshift estimator \citep[\cmnn,][]{Graham:17} uses a training set of galaxies with known redshifts that has equivalent or better photometry than the test set in terms of quality and filter coverage.
For each galaxy in the test set, \cmnn\ identifies a colour-matched subset of training galaxies using a threshold in the Mahalanobis distance $D_M = \sum_{j}^{N_{\rm colours}} (c^{\rm train}_{j} - c^{\rm test}_{j})^{2} / \delta c_{\rm test}^2$ in the space of available colours $c$, with colour measurement errors $\delta c_{\rm test}$ and $N_{\rm colours} = 5$ colors $j$ defined by the $ugrizy$ filters, which defines the set of colour-matched neighbours based on a value of the percent point function (PPF).
As an example, for $N_{\rm{filt}}=5$ with PPF$=0.95$, $95\%$ of all training galaxies consistent with the test galaxy will have $D_M < 11.07$.
Undetected bands are dropped, thereby reducing the effective $N_{\rm{filt}}$ for that galaxy.
The \pzpdf\ of a given test set galaxy is the normalized distribution of redshifts of its colour-matched subset of training set galaxies.

Here, we make two modifications to the implementation of \citet{Graham:17} to comply with the controlled experimental conditions.
First, we do not impose nondetections on galaxies fainter than the expected \lsst\ 10-year limiting magnitude or bright enough to saturate with \lsst's CCDs, instead using all of the photometry for the DC1 test and training sets.
Second, we apply the initial colour cut to the training set before calculating the Mahalanobis distance in order to accelerate processing and use a magnitude pseudo-prior as in \citet{Graham:17}, but for both we use cut-off values corresponding to the DC1 training set galaxies' colours and magnitudes.

We make an additional adaptation to enable the \cmnn\ algorithm to yield accurate \pzpdf s for all galaxies, as the original \citet{Graham:17} algorithm is optimized for \pz\ point estimates and is susceptible to less accurate \pzpdf s for bright galaxies or those with few matches in colour-space.
We use PPF$=0.95$ rather than PPF$=0.68$ to generate the subset of colour-matched training galaxies, whose redshifts are weighted by their inverse Mahalanobis distances of the when composing the \pzpdf\ rather than weighting all colour-matched training galaxies equally.
Additionally, when the number of colour-matched training set galaxies is less than 20, the nearest 20 neighbours in color-space are used instead, and the output \pzpdf\ is convolved with a Gaussian kernel of variance $\sigma_{\rm train}^{2}({\rm PPF}_{20}/0.95)^2 -1$ to account foe the corresponding growth of the effective PPF to include 20 neighbors.

\subsubsection{FlexZBoost}
\label{sec:flexzboost}
%(Ann Lee, Rafael Izbicki, Taylor Pospisil, Peter Freeman)

\flexzboost \footnote{\url{https://github.com/tpospisi/flexcode};  \url{https://github.com/rizbicki/FlexCoDE} \label{flexzboost_github}} \citep{Izbicki:17} is built on \texttt{FlexCode}, a general-purpose methodology for converting any conditional mean point estimator of $z$ to a conditional density estimator $p(z \vert \x) \equiv f(z \vert \x)$, where $\x$ here represents our photometric covariates and errors.
\flexzboost\ expands the unknown function $f(z \vert \x) = \sum_{i}\beta_{i}(\x)\phi_{i}(z)$ using an orthonormal basis $\{\phi_{i}(z)\}_{i}$.
By the orthogonality property, the expansion coefficients $\beta_{i}(\x) = \mathbb{E}\left[\phi_i(z)|\x\right] \equiv \int f(z \vert \x) \phi_{i}(z) dz$ are thus conditional means.
The expectation value $\mathbb{E}\left[\phi_i(z) \vert \x\right]$ of the expansion coefficients conditioned on the data is equivalent to the regression of the space of possible redshifts on the space of possible photometry.
Thus the expansion coefficients $\beta_{i}(\x)$ can be estimated from the data via regression to yield the conditional density estimate $\widehat{f}(z \vert \x)$.

In this paper, we used \texttt{xgboost} \citep{Chen:16} for the regression; it should however be noted that \texttt{FlexCode-RF}\footref{flexzboost_github}, based on Random Forests, generally performs better for smaller datasets.
As our basis $\phi_{i}(z)$, we choose a standard Fourier basis.
The two tuning parameters in our \pzpdf\ estimate are the number $I$ of terms in the series expansion and an exponent $\alpha$ that we use to sharpen the computed density estimates $\widetilde{f}(z \vert \x) \propto \widehat{f}(z \vert \x)^{\alpha}$.
Both $I$ and $\alpha$ were chosen in an automated way by minimizing the weighted $L_2$-loss function \citep[Eq. 5 in][]{Izbicki:17} on a validation set comprised of a randomly selected 15\% of the DC1 training set.
While \texttt{FlexCode}'s lossless native encoding stores each \pzpdf\ using the basis coefficients $\beta_{i}(\x)$, we discretized the final estimates into 200 linearly-spaced redshift bins $0 < z < 2$ to match the consistent output format of the experimental conditions.

%\subsubsection{Frankenz}
%\label{sec:frankenz}
%(seeking volunteers)
%
%\red{\textsc{Frankenz}\footnote{\url{https://github.com/joshspeagle/frankenz}} \cite{Speagle:frankenz} is...}
%

\subsubsection{GPz}
\label{sec:gpz}
%(Ibrahim Almosallam)

\gpz \footnote{\url{https://github.com/OxfordML/GPz}} \citep{Almosallam:16a,Almosallam:15b} is a sparse Gaussian process based code, a scalable approximation of full Gaussian Processes \citep{Rasmussen:06}, that produces input-dependent variance estimates corresponding to heteroscedastic noise.
The model assumes a Gaussian posterior probability $p(z \vert \x) = \mathcal{N}\left(z \vert \mu(\x), \sigma(\x)^{2}\right)$ of the output redshift $z$ given the input photometry $\x$.
The mean $\mu(\x)$ and the variance $\sigma(\x)^{2}$ are modeled as functions $f(\x) = \sum_{i=1}^{m}w_{i}\phi_{i}(\x)$ linear combinations of $m$ basis functions $\left\{\phi_{i}(\x)\right\}_{i=1}^{m}$ with associated weights $\left\{w_{i}\right\}_{i=1}^{m}$.
% Basis function models, for specific classes of basis functions such as the sigmoid or the squared exponential, have the advantage of being universal approximators, i.e. there exist a function of that form that can approximate any function, with mild assumptions, to any desired degree of accuracy.
The details on how to learn the parameters of the model and the hyper-parameters of the basis functions are described in \citet{Almosallam:15b}.
\gpz's variance estimate is composed of a model uncertainty term corresponding to sparsity of the training set photometry and a noise uncertainty term encompassing noisy photometric observations, enabling quantification of any need for more representative or more precise training samples.
\gpz\ may also weight training set samples by importance according to $|z_{\rm{spec}} - z_{\rm{phot}}| / (1+z_{\rm{spec}})$ to minimize the normalized \pz\ point estimate error, however, this function may be adapted to \pzpdf s, pressuring the model to dedicate more resources to test set galaxies that are not well-represented in the training set.

To smooth the long tail in the distribution of magnitude errors, we use the log of the magnitude errors, improving numerical stability and eliminating the need for constraints on the optimization process.
% We use principal component analysis (PCA) to decorrelate the data
% WHAT IS BEING DECORRELATED HERE?
Unobserved magnitudes $x_{\rm u} = \mu_{\rm u} + \Sigma_{\rm uo}\Sigma_{\rm oo}^{-1}(x_{\rm o} - \mu_{\rm o})$ were imputed from observed magnitudes $x_{\rm o}$ and the training set mean $\mu$ and covariance $\Sigma$ using a linear model.
This is the optimal expected value of the unobserved variables given the observed ones under the assumption that the distribution is jointly Gaussian; note that this reduces to a simple average if the covariates are independent with $\Sigma_{\rm uo} = 0$.
We reserved for validation 20\% of the training set and used the Variable Covariance option in \textsc{GPz} with 200 basis functions, neglecting to apply cost-sensitive learning options.

\subsubsection{METAPhoR}
\label{sec:metaphor}
%(Stefano Cavuoti, Massimo Brescia, Giuseppe Longo)

\textsc{METAPhoR} \citep[Machine-learning Estimation Tool for Accurate Photometric Redshifts,][]{Cavuoti:17} is based on the Multi Layer Perceptron with Quasi Newton Algorithm (MLPQNA) with the least square error model and Tikhonov $L_{2}$-norm regularization \citep{Hofmann:18}.
%, already validated on photo-$z$'s in several cases \citep{de_Jong:17,Cavuoti:17b,Cavuoti:15,Brescia:14,Brescia:13,Biviano:13}.
\Pzpdf s are generated by running $N$ trainings on the same training set, or $M$ trainings on $M$ different random samplings of the training set.
Upon regression of the test set, the photometry $m_{ij}$ of each test set galaxy $j$ in filter $i$ is perturbed according to $m_{ij}' = m_{ij} + \alpha_{i} F_{ij} \epsilon$ in terms of the standard normal random variable $\epsilon \sim \mathcal{N}(0, 1)$, a multiplicative constant $\alpha_{i}$ permitting accommodation of multi-survey photometry, and a bimodal function $F_{ij}$ composed of a polynomial fit of the mean magnitude errors on the binned bands plus a constant term representing the threshold below which the polynomial's noise contribution is negligible \citep{Brescia:18}.
% At a higher level, the pipeline mainly consists of three modules: (i) \textit{data pre-processing}, including a catalogue cross-matching sub-module \citep[based on the tool C3, ][]{Riccio:17}, a sub-module for photometric evaluation and error estimation of the multi-band catalogue used as Knowledge Base (KB), and a sub-module dedicated to the perturbation of the photometric KB, propaedeutic to the PDF estimation; (ii) \textit{photo-$z$ prediction}, which is the training/validation/test phase, producing the photo-$z$'s point estimates, based on a pre-selected ML method; (iii) \textit{PDF estimation}, specifically designed to calculate the PDF of the photo-$z$ estimation errors.
% The last module includes also a post-processing tool, providing some statistics on the produced point estimates and PDFs.

In this work, we used a hierarchical KNN to replace nondetections with values based on their neighbors.
The usual cross-validation step was also omitted for this study.

\subsubsection{SkyNet}
\label{sec:skynet}
%(J. Cohen-Tanugi and Hugo Tranin)

\skynet \footnote{\url{http://ccpforge.cse.rl.ac.uk/gf/project/skynet/}} \citep{Graff:14} employs a neural network based on a second order conjugate gradient optimization scheme \citep[see][for further details]{Graff:14}. %It has been used efficiently for redshift PDF estimates \citep{Sanchez:14,Bonnett:15,Bonnett:16}.
The neural network is configured as a standard multilayer perceptron with three hidden layers and one input layer with 12 nodes corresponding to the 6 photometric magnitudes and their measurement errors.
We use \skynet\ as a regressor for \pz\ point estimation and as a classifier for \pzpdf\ estimation.

The regressor used a standard \chisq\ error function with a single linear node as the output layer and 10 nodes with a $\tanh$ activation function for each hidden layer.
The classifier used a cross-entropy error function with a 20:40:40 node (all rectified linear units) architecture for each hidden layer and an output layer of 200 nodes corresponding to 200 bins for the PDF, with a softmax activation function to enforce the normalization condition that the probabilities sum to unity.
While previous implementations of the code \citep[see Appendix C.3 of~][]{Sanchez:14,Bonnett:15} implement a sliding bin smoothing, no such procedure was used in this study.

We pre-whitened the data by pegging the magnitudes to (45,45,40,35,42,42) and errors to (20,20,10,5,15,15) for $ugrizy$ filters, respectively.
% NO CLUDE WHAT THIS MEANS
To avoid over-fitting, $30\%$ of the training set was reserved for validation, and training was halted as soon as the error rate began to increase on the validation set.
The weights were randomly initialized based on normal sampling.
% WHAT WEIGHTS?

\subsubsection{TPZ}
\label{sec:tpz}
%(Erfan Nourbakhsh)

\textsc{TPZ}\footnote{\url{https://github.com/mgckind/MLZ}} \citep[Trees for Photo-$z$,][]{Carrasco_Kind:13,Carrascokind:14} is a parallel machine learning algorithm that generates photometric redshift PDFs using prediction trees and random forest techniques.
The code recursively splits the input data (i.~e.~the training sample), into two branches, one after another, until a terminal leaf is created that meets a termination criterion (e.~g.~a minimum leaf size or a variance threshold).
Bootstrap samples from the training data and associated errors are used to build a set of prediction trees.
In order to minimize correlation between the trees, the data is divided in such a way that the highest information gain among the random subsample of features is obtained at every point.
The regions in each terminal leaf node corresponds to a specific subsample of the entire data that possesses similar properties.

The training data is examined before running TPZ.
Since TPZ does not handle non-detections (magnitudes flagged as 99.0), we replace these values with an approximation of the $1\sigma$ detection threshold, i.~e.~a signal to noise ratio of 1 in terms of magnitude uncertainty using the equation $dm = 2.5 ~ \log ( 1 + N/S )$ where $dm \sim 0.7526 ~ mag$ for $N/S=1$.
That is, for each band, we replace the non-detection with the magnitude corresponding to the error of 0.7526 from the error model forecasted for 10-year LSST data.
The Out-of-Bag \citep{Breiman:84,Carrasco_Kind:13} cross-validation technique is used within TPZ to evaluate its predictive validity and determine the relative importance of the different input attributes.
We employed this information to calibrate our algorithm.

In the present work, the LSST magnitudes $u,~g,~r,~i$ and colours $u-g,~g-r,~r-i,~i-z,~z-y$ and their associated errors are used in the process of growing 100 trees with a minimum leaf size of 5 (the $z$ and $y$ magnitudes did not show significant correlation with the redshift in our cross-validation, so we did not use them when constructing our trees).
We partitioned our redshift space into 200 bins and smoothed each individual PDF with a smoothing scale of twice the bin size.
%there was a typo here that had 100 bins, I double checked, it is 200 in the final run.

\subsubsection{Delight}
\label{sec:delight}
%(John Soo)

\delight \footnote{\url{https://github.com/ixkael/Delight}} \citep{Leistedt:17} infers \pz s with a data-driven model of latent SEDs and a physical model of photometric fluxes as a function of redshift.
Delight models the underlying latent SEDs as a linear combination of a set of pre-defined template SEDs, plus zero mean Gaussian processes with factorized kernels.
Generally, machine learning methods rely on representative training data with similar band passes, while template based methods rely on a complete library of templates based on physical models constructed.
\textsc{Delight} is constructed in attempt to combine the advantages and eliminate the disadvantages of both template-based and machine learning algorithms: it constructs a large collection of latent SED templates (or physical flux-redshift models) from training data, with a template SED library as a guide to the learning of the model.
The advantage of \textsc{Delight} is that it neither needs representative training data in the same photometric bands, nor does it need detailed galaxy SED models to work.

This conceptually novel approach is done by using Gaussian processes operating in flux-redshift space.
The posterior distribution on the redshift of a target galaxy is obtained via a pairwise comparison with training galaxies,

\begin{equation}
p(z|\mathbf{\hat{F}}) \approx \sum_i p(\mathbf{\hat{F}}|z,t_i)\, p(z|t_i)p(t_i),
\end{equation}
\noindent where $p(z|t_i)p(t_i)$ captures prior information about the redshift distributions and abundances of the galaxies, with $t_i$ denoting the galaxy template; while $p(\mathbf{\hat{F}}|z,t_i)$ is the posterior of noisy flux $\mathbf{\hat{F}}$ at redshift $z$.
For each training-target pair, $p(\mathbf{\hat{F}}|z,t_i)$ is evaluated as follows:
\begin{equation} \label{eq:delight_noisy}
p(\mathbf{\hat{F}}|z,t_i) = \int p(\mathbf{\hat{F}}|\mathbf{F})\, p(\mathbf{F}|z,z_i,\mathbf{\hat{F}}_i)\, d\mathbf{F},
\end{equation}
where $p(\mathbf{\hat{F}}|\mathbf{F})$ is the likelihood function, it compares the noisy real flux $\mathbf{\hat{F}}$ with the noiseless flux $\mathbf{F}$ obtained from the linear combination of template models, carefully constructed to account for model uncertainties and different normalization of the same SED; while $p(\mathbf{F}|z,z_i,\mathbf{\hat{F}}_i)$ is the prediction of flux at a different redshift $z$ with respect to the training object with redshift $z_i$ and flux $\mathbf{\hat{F}}_i$. Eq.~\ref{eq:delight_noisy} is essentially the probability that the training and the target galaxies having the same SED but at a different redshift.
The flux prediction $p(\mathbf{F}|z,z_i,\mathbf{\hat{F}}_i)$ of the training galaxy at redshift $z$ is modeled via a Gaussian process,

\begin{equation} \label{eq:delight_gp}
F_b \sim \mathcal{GP}\left( \mu^F,k^F \right),
\end{equation}
\noindent with mean function $\mu^F$ and kernel $k^F$, both imposed to capture expected correlations resulting from the known underlying physics (i.e., fluxes resulting from observing SEDs through filter response, and the SEDs being redshifted).
The reader should refer to \citet{Leistedt:17} for further details.

In this study, all $100$ ordered Buzzard templates, as described in Section~\ref{sec:buzztemplates}, were used in \textsc{Delight}, and the Gaussian process was trained using the provided training sample.
Photometric uncertainties from the inputs are propagated into the code, while non-detections for each band are set to the mean of the respective bands.
The default settings of \textsc{Delight} were used, with the exception that the PDF bins were set to be linearly-spaced rather than logarithmic. In this study a flat prior in magnitude/type is assumed.

\subsection{trainZ: a straw-person \pz\ estimator}
\label{sec:method:trainz}

In addition to the main photo-$z$ algorithms described above we also include a very simple method as a pathological example.
For \trainz, as we will we call this simple estimator, we well define $p(z)$ as simply:
\begin{equation}
p(z) = \frac{1}{N_{ \mathrm train}}\sum_{\mathrm i=1}^{N_{\mathrm train}}z_{\mathrm train}
\end{equation}
That is, we simply set the redshift PDF of every galaxy equal to the normalized $N(z)$ of the training sample.
This estimator is essentially a k nearest-neighbour estimator with k equal to the number of galaxies in the training sample.
As the training sample is drawn from the same underlying distribution as the test sample, modulo small deviations due to sample size, the quantiles of the training and test distributions should be identical, modulo fluctuations due to finite sample size.
This is a wildly unrealistic estimator, as it assigns all galaxies, no matter their apparent magnitude, colour, or true redshift, the same redshift PDF, and is thus uninformative at the level of individual object redshifts, but is designed to perform very well for the ensemble of all objects.
If the training set was not representative, this estimator would produce biased results, and any attempts to break up the sample into tomographic bins will fail, as every galaxy has an identical $p(z)$.
We will discuss this method and cautions relative to metrics in Section~\ref{sec:caution}.


\section{Analysis}
\label{sec:metrics}

The goal of this study is to evaluate the degree to which \pzpdf s of each method can be trusted for a generic analysis.
The overloaded ``$p(z)$'' is a widespread abuse of notation that obfuscates this goal, so we dedicate attention to dismantling it here.

Galaxies have redshifts $z$ and photometric data $d$ drawn from a joint probability space $p(z, d)$ in nature, and each observed galaxy $i$ has a \textit{true posterior \pzpdf}\ $p(z \vert d_{i})$.
There are a number of metrics that can be used to test the accuracy of a \pz\ posterior as an estimator of a true \pz\ posterior if the true \pzpdf\ is known.
However, the true \pzpdf\ of the observed data is not accessible, as existing mock catalogues produce redshift-photometry pairs $(z, d)$ by a deterministic algorithm that does not correspond to a joint probability density from which one can take samples.
In these cases there is no ``true PDF'' for an individual object
\footnote{\boldblue{While a discrete approximation to the true $p(z,d)$ is possible by sampling the local neighbourhood of parameter space in large datasets with methods like a nearest neighbor or conditional density estimate, we do not make an explicit computation of such an approximate distribution.  Rather, we note that the CMNN and FlexZBoost algorithms are specific implementations of such algorithms which can be used as examples of such approximations in the absence of knowledge of the true $p(z,d)$.}}, and most measures of PDF fidelity will necessarily be restricted to probing the quality of the ensemble of \pzpdf s.
(See \S,\ref{sec:futureexperiments} for a discussion of how one might circumvent this limitation.)

Before describing the metrics appropriate to the DC1 data set, we outline the philosophy behind our choices.
A \pzpdf\ estimator derived by method $H$ must be understood as a posterior probability distribution
\begin{equation}
  \label{eq:pzpdf}
\hat{p}^{H}_{i}(z) \equiv p(z \vert d_{i}, I_{D}, I_{H}),
\end{equation}
conditioned not only on the photometric data $d_{i}$ for that galaxy but also on parameters encompassing prior information $I_{D}$ shared, in our experiment, among all \pzpdf\ codes and $I_{H}$ that will differ depending on the method $H$ used to produce it.
To be concrete, $I_{D}$ takes the form of a training set for the machine learning codes and a template library for the model fitting codes.

The interpretation of the information $I_{H}$ is more subtle.
This investigation is built upon the knowledge that two codes taking the same approach, among choices of model fitting or machine learning, are nonetheless expected to yield different results even if they take the same external prior information $I_{D}$.
$I_{H}$ represents the projection of the code's architecture onto the estimated posteriors over redshift, specific to each code, and even the tunable parameters or random seeds of a specific run of a code with a random component.
We refer to $I_{H}$ as the \textit{implicit prior}, in contrast with the training set or template library provided to a given code explicitly by the researcher.  In simple terms, the implicit prior is the collection of the many different assumptions, coding choices, algorithm selections, and other implementation details that are specific to each code, the ensemble of which results in differing estimates of redshift when combined with the data and prior information in common to all codes.

The presence of the implicit prior in some sense makes a direct comparison of \pzpdf s produced by different methods impossible; even if they share the same external prior information $I_{D}$, by definition they cannot be conditioned on the same assumptions $I_{H}$, otherwise they would not be distinct methods at all.
In this study, we isolate the effect of differences in prior information $I_{H}$ specific to each method by using a single training set $I_{D}^{\mathrm{ML}}$ for all machine learning-based codes and a single template library $I_{D}^{\mathrm{T}}$ for all template-based codes.
These sets of prior information are carefully constructed to be representative and complete, so we have $I_{D} \equiv I_{D}^{\mathrm{ML}} \equiv I_{D}^{\mathrm{T}}$ for every method $H$.
Under this assumption, a ratio of posteriors of codes is in effect a ratio of the implicit posteriors $p(z \vert d_{i}, I_{H'})$ since the external prior information $I_{D}$ is present in the numerator and denominator.
Thus comparisons of $\hat{p}_{i}^{H}(z)$ isolate the effect of the method used to obtain the estimator, which should enable interpretation of the differences between estimated PDFs in terms of the specifics of the method implementations.

The exact implementation of the metrics theoretically depends on the parametrization of the \pzpdf s, which may differ across codes and can affect the precision of the estimator \citep{Malz:2018}.
Even considering a single method under the same parametrization, such as the 200-bin $0 < z < 2$ piecewise constant function used here, the exact bin definitions must affect the result.
The piecewise constant format is chosen because of its established presence in the literature, and the choice of 200 bins was motivated by the approximate number of columns expected to be available for storage of \pzpdf s for the final \lsst\ Project tables.\footnote{See, e.~g.~the \lsst\ Data Products Definition Document, available at: \url{https://ls.st/dpdd}}
We will discuss the choice of \pzpdf\ parameterization further in Section~\ref{sec:discussion}.

This analysis is conducted using the \qp\footnote{\url{http://github.com/aimalz/qp/}}\ software package \citep{Malz:qp} for manipulating and calculating metrics of univariate PDFs.
We present the metrics of \pzpdf s that address our goals in the sections below.
Section~\ref{sec:qualmet} outlines aggregate metrics of a catalogue of \pzpdf s, and Section~\ref{sec:CDE_loss} presents a metric of individual \pzpdf s in the absence of true \pzpdf s.
Those seeking a connection to previous comparison studies will find metrics of redshift point estimate reductions of \pzpdf s in Appendix~\ref{sec:pointmetrics} and metrics of a science-specific summary statistic heuristically derived from \pzpdf s in Appendix~\ref{sec:moments}.

\subsection{Metrics of \pzpdf \ ensembles}
\label{sec:qualmet}

Because \lsst's \pzpdf s will be used for many scientific applications, some of which require each individual catalogue entry to be accurate, we consider several metrics that probe the population-level performance of the \pzpdf s.
As we have the true redshifts but not true \pzpdf s for comparison, we remind the reader of the Cumulative Distribution Function (CDF)
\begin{equation}
  \label{eq:cdf}
  \mathrm{CDF}[f, q] \equiv \int_{-\infty}^{q} f(z) dz,
\end{equation}
of a generic univariate PDF $f(z)$, which is used as the basis for several of our metrics.
We describe metrics based on the CDF in Section~\ref{sec:qqpit} and metrics of summary statistics thereof in Section~\ref{sec:summqqpit}.

\subsubsection{CDF-based metrics}
\label{sec:qqpit}

A quantile of a distribution is the value $q$ at which the CDF of the distribution is equal to $Q$; percentiles and quartiles are familiar examples of linearly spaced sets of 100 and 4 quantiles, respectively.
The quantile-quantile (QQ) plot serves as a graphical visualization for comparing two distributions, where the quantiles of one distribution are plotted against the quantiles of the other distribution, providing an intuitive way to qualitatively assess the consistency between an estimated distribution and a true distribution.
The closer the QQ plot is to diagonal, the closer the match between the distributions.

The probability integral transform (PIT)
\begin{align}
\label{eq:pit}
\mathrm{PIT} &\equiv \mathrm{CDF}[\hat{p}, z_{\mathrm true}]
\end{align}
is the CDF of a \pzpdf\ evaluated at its true redshift, and the distribution of PIT values probes the average accuracy of the \pzpdf s of an ensemble of galaxies.
The distribution of PIT values is effectively the derivative of the QQ plot.
A catalogue of accurate \pzpdf s should have a PIT distribution that is uniform $U(0,1)$, and deviations from flatness are interpretable: overly broad \pzpdf s induce underrepresentation of the lowest and highest PIT values, whereas overly narrow \pzpdf s induce overrepresentation of the lowest and highest PIT values.
Catastrophic outliers with a true redshift outside the support of its \pzpdf\ have $\mathrm{PIT} \approx 0$ or $\mathrm{PIT} \approx 1$.

The PIT distribution has been used to quantify the performance of \pzpdf\ methods in the past \citep[e.~g.~][]{Bordoloi:10,Polsterer:16,Tanaka:17}.
\citet{Tanaka:17} use the histogram of PIT values as a diagnostic indicator of overall code performance, while \citet{Freeman:17} independently define the PIT and demonstrate how its individual values may be used both to perform hypothesis testing (via, e.~g.~ the KS, CvM, and AD tests; see below) and to construct QQ plots.
Following Kodra \& Newman (in prep.) we define the PIT-based catastrophic outlier rate as the fraction of galaxies with $\mathrm{PIT} < 0.0001$ or $\mathrm{PIT} > 0.9999$, which should total 0.0002 for an ideal uniform distribution.

\subsubsection{Summary statistics of CDF-based metrics}
\label{sec:summqqpit}

We evaluate a number of quantitative metrics derived from the visually interpretable QQ plot and PIT histogram, built on the Kolmogorov-Smirnov (KS) statistic
\begin{equation}
  \label{eq:ks}
  \mathrm{KS} \equiv \max_{z} \left( \left| \mathrm{CDF}[\hat{f}, z] - \mathrm{CDF}[\tilde{f}, z] \right| \right),
\end{equation}
interpretable as the maximum difference between the CDFs of the empirical distribution of PIT values for the test sample $\hat{f}(z)$ and a reference distribution $\tilde{f}(z)$, in this case $U(0,1)$, for the ideal distribution of PIT values.
We also consider two variants of the KS statistic.
A cousin of the KS statistic, the Cramer-von Mises (CvM) statistic
\begin{equation}
\label{eq:cvm}
  \mathrm{CvM}^{2} \equiv \int_{-\infty}^{+\infty} \big(\mathrm{CDF}[\hat{f}, z] - \mathrm{CDF}[\tilde{f}, z]\big)^2 \mathrm{d}\mathrm{CDF}[\tilde{f}, z]
\end{equation}
is the mean-squared difference between the CDFs of an approximate and true PDF.
The Anderson-Darling (AD) statistic
\begin{equation} \label{eq:ad}
  \mathrm{AD}^2 \equiv N_{tot}\int_{-\infty}^{+\infty} \frac{\big(\mathrm{CDF}[\hat{f}, z] - \mathrm{CDF}[\tilde{f}, z]\big)^2} {\mathrm{CDF}[\tilde{f}, z] (1 -\mathrm{CDF}[\tilde{f}, z])} \mathrm{d}\mathrm{CDF}[\tilde{f}, z]
\end{equation}
is a weighted mean-squared difference featuring enhanced sensitivity to discrepancies in the tails of the distribution.
In anticipation of a substantial fraction of galaxies having PIT of 0 or 1, a consequence of catastrophic outliers, we evaluate the AD statistic with modified bounds of integration $(0.01, 0.99)$ to exclude those extremes in the name of numerical stability.

\subsection{Conditional Density Estimate (CDE) Loss: a metric of individual \pzpdf s}
\label{sec:CDE_loss}

The \buzz\ simulation process precludes testing the degree to which samples from our \pz\ posteriors reconstruct the space of $p(z, \mathrm{data})$.
To the knowledge of the authors, there is only one metric that can be used to evaluate the performance of individual \pzpdf\ estimators in the absence of true \pz\ posteriors.
The conditional density estimation (CDE) loss is an analogue to the familiar root-mean-square-error used in conventional regression, defined as
\begin{equation}
  \label{eq:cde-loss}
  L(f, \widehat{f}) \equiv \int \int (f(z \vert \x) - \widehat{f}(z \vert \x))^{2} \mathrm{d}z \mathrm{d}P(\x) ,
\end{equation}
where $f(z \vert \x)$ is the true \pzpdf\ that we do not have and $\widehat{f}(z \vert \x)$ is an estimate thereof, in terms of the photometry $\x$.
(See Section~\ref{sec:flexzboost} for a review of the notation.)
We estimate the CDE loss via
\begin{equation}
  \label{eq:estimated-cde-loss}
  \hat{L}(f, \widehat{f}) = \mathbb{E}_\X \left[\int \widehat{f}(z \mid \X)^{2} dz\right] - 2 \mathbb{E}_{\X, Z}\left[\widehat{f}(Z \mid \X)\right] + K_{f},
\end{equation}
where the first term is the expectation value of the \pz\ posterior with respect to the marginal distribution of the photometric covariates $\X$, the second term is the expectation value with respect to the joint distribution of $\X$ and the space $Z$ of all possible redshifts, and the third term $K_{f}$ is a constant depending only upon the true conditional densities $f(z \vert \x)$\xsout{We may estimate these expectations empirically on the test or validation data} \citep[Eq.~7 in][]{Izbicki:17b}. \xsout{without knowledge of the true densities.}
\boldblue{One of the most powerful features of the CDE loss is the ability to estimate the loss function up to a constant even in the absence of knowledge of the true underlying distribution (see Eq.~7 and accompanying discussion in \citet{Izbicki:17} for more details).  This feature enables a quantitative comparison of our estimation methods with our current dataset where we lack access to the true $f(z \vert \x)$.}


\section{Results}
\label{sec:results}
%(Sam Schmidt, Bryce Kalmbach, Johann Cohen Tanugi, Rongpu Zhou, Alex Malz)

We begin with a demonstrative visual inspection of the \pzpdf s produced by each code for individual galaxies.
Figure~\ref{fig:pz_examples} shows the \pzpdf s for four galaxies chosen as examples of \pzpdf\ archetypes: a narrow unimodal PDF, a broad unimodal PDF, a bimodal PDF, and a multimodal PDF.
We reiterate that under our idealized experimental conditions, differences between codes are the isolated signature of the implicit prior due to the method by which the \pzpdf s were derived.

\begin{figure*}
%\centering
\includegraphics[width=0.49\textwidth]{fig/pz_12codes_261931_noseaborn_crop.jpg}\includegraphics[width=0.49\textwidth]{fig/pz_12codes_471167_noseaborn_crop.jpg}\\
\includegraphics[width=0.49\textwidth]{fig/pz_12codes_713178_noseaborn_crop.jpg}\includegraphics[width=0.49\textwidth]{fig/pz_12codes_982747_noseaborn_crop.jpg}
\caption{The individual \pzpdf s (blue) distributions produced by the twelve codes (small panels) on four exemplary galaxies' photometry (large panels) with different true redshifts (red).
The \pzpdf s of all codes share some features for the example galaxies due to physical color degeneracies and photometric errors: tight unimodal $p(z)$ (upper left), broad unimodal $p(z)$ (upper right), bimodal $p(z)$ (lower right), and complex/multimodal $p(z)$ (lower left).
The diverse algorithms and implementations induce differences in small-scale structure and sensitivity to physical systematics.}
\label{fig:pz_examples}
\end{figure*}

The most striking differences between codes are due to small-scale features induced by the interaction between the shared piecewise constant parameterization of $200$ bins $0 < z < 2$ of Section~\ref{sec:metrics} and the smoothing conditions or lack thereof in each algorithm.
The $\rm{d}z = 0.01$ redshift resolution is sufficient to capture the broad peaks of faint galaxies' \pzpdf s with large photometric errors but is too broad to resolve the narrow peaks for bright galaxies' \pzpdf s with small photometric errors.
This observation is consistent with the findings of \citet[]{Malz:qp} that the piecewise constant parameterization underperforms in the presence of small-scale structures.

However, the shared small-scale features of \annz, \metaphor, \cmnn, and \skynet\ are a result of various weighted sums of the limited number of training set galaxies with colors similar to those of the test set galaxy in question, with behavior closer to classification than regression in the case of \annz.
The settings used on \gpz\ in this work forced broadening of the single Gaussian to cover the multimodal redshift solutions of the other codes.
%Interestingly, while \textsc{ANNz2} shows an abundance of small scale structure in individual $p(z)$ measurements (see Fig.~\ref{fig:pz_examples}), the summed $\hat{N}(z)$ is rather smooth, where the small scale features average out.  This is not the case for the two other codes that show an abundance of substructure in their individual $p(z)$: both \textsc{CMNN} and \textsc{SkyNet} show small scale features both in $p(z)$ and $\hat{N}(z)$.
%In contrast, FlexZBoost, for example, can return estimates on any grid without compression errors as it’s a basis expansion method where only the expansion coefficients need to be stored.
%Codes with a native output format other than the shared piecewise constant binning scheme (or one that can be losslessly converted to it) may suffer from loss of information when converting to it, which could artificially favor some codes over others in a limited number of cases, for example bright galaxies with very narrow $p(z)$ where the true peak falls between grid points.  We will discuss PDF storage in Section~\ref{sec:discussion}.
%Furthermore, the fidelity of photo-$z$ interim posteriors in this format varies with the quality of the photometry.
%Switching to a quantile based parameterization may be more costly computationally, for example template-based codes would need to test more grid point in order to resolve the quantiles for bright galaxies.  However, the computational time for template based codes scales roughly linearly with the number of grid points, so this may be a reasonable thing to implement.
% We will discuss this further in Section~\ref{sec:discussion}.
% \red{someone review this statement to make sure that I'm saying this correctly!}

\subsection{Performance on \pzpdf\ ensembles}
\label{sec:pitqq}

Figure~\ref{fig:pitqq} shows a histogram of PIT values, QQ plot, and QQ difference plot relative to the ideal diagonal, showcasing the biases and trends in the average accuracy of the \pzpdf s for each code.
The high QQ values (more high than low PIT values) of \bpz, \cmnn, \delight, \eazy, and \gpz\ indicate \pzpdf s biased low, and the low QQ values (more low than high PIT values) of \skynet\ and \tpz\ indicate \pzpdf s biased high.

\begin{figure*}
\centering
\includegraphics[width=0.74\textwidth]{fig/PITANDQQplot_12codes_crop.jpg}
\caption{The QQ plot (red) and PIT histogram (blue) of the \pzpdf\ codes (panels) along with the ideal QQ (black dashed diagonal) and ideal PIT (gray horizontal) curves, as well as a difference plot for the QQ difference from the ideal diagonal (lower inset).
The twelve codes exhibit varying degrees of four deviations from perfection: an overabundance of PIT values at the centre of the distribution indicate a catalogue of overly broad \pzpdf s, an excess of PIT values at the extrema indicates a catalogue of overly narrow \pzpdf s, catastrophic outliers manifest as overabundances at PIT values of 0 and 1, and asymmetry indicates systematic bias, a form of model misspecification.}
\label{fig:pitqq}
\end{figure*}

The PIT histograms of \delight, \cmnn, \skynet, and \tpz\ feature an underrepresentation of extreme values, indicative of overly broad \pzpdf s, while the overrepresentation of extreme values for for \metaphor indicate overly narrow \pzpdf s.
These five codes in particular have a free parameter for bandwidth, which may be responsible for this vulnerability, in spite of the opportunity for fine-tuning with perfect prior information.
\flexzboost's ``sharpening'' parameter (described in Section~\ref{sec:flexzboost}) played a key role in diagonalizing the QQ plot, indicating a common avenue for improvement in the approaches that share this type of parameter.
On the other hand, the three purely template-based codes, \bpz, \eazy, and \lephare, do not exhibit much systematic broadening or narrowing, which may indicate that complete template coverage effectively defends from these effects.

\begin{table}
\setlength{\tabcolsep}{2pt}
\centering
\caption{The catastrophic outlier rate as defined by extreme PIT values.
We expect a value of 0.0002 for a proper Uniform distribution.
An excess over this small value indicates true redshifts that fall outside the non-zero support of the $p(z)$.}
\label{tab:pitoutlier}
\begin{tabular}{lc}
\hline
\hline
\Pz\ Code & fraction PIT$<10^{-4}$ or $>$0.9999\\
\hline
\annz       & 0.0265\\
\bpz        & 0.0192\\
\delight    & 0.0006\\
\eazy       & 0.0154\\
\flexzboost & 0.0202\\
\gpz        & 0.0058\\
\lephare    & 0.0486\\
\metaphor   & 0.0229\\
\cmnn       & 0.0034\\
\skynet     & 0.0001\\
\tpz        & 0.0130\\
\hline
\trainz     & 0.0002\\
\end{tabular}
\end{table}

Though the spikes in the first and last bin of the PIT histogram were cut off in Figure~\ref{fig:pitqq} for visualization, the catastrophic outlier rates are provided in Table~\ref{tab:pitoutlier}.
As expected, \trainz\ achieves precisely the 0.0002 value expected of an ideal PIT distribution.
\annz, \flexzboost, \lephare, and \metaphor\ have notably high catastrophic outlier rates $> 0.02$, exceeding 100 times the ideal PIT rate, meriting further investigation.

\begin{figure*}
\centering
\includegraphics[width=0.74\textwidth]{fig/KSvsCvMvsAD_PIT_withnull_jpg.jpg}
\caption{A visualization of the Kolmogorov-Smirnoff (KS, blue diamond), Cramer-von Mises (CvM, black star), and Anderson-Darling (AD, red asterisk) statistics for the PIT distributions.
There is generally good agreement between these statistics, with differences corresponding to the codes with outstanding catastrophic outlier rates, a reflection in the differences in how each statistic weights the tails of the distribution.}
\label{fig:pit_stats}
\end{figure*}

Figure~\ref{fig:pit_stats} displays the values of the KS, CvM, and AD test statistics between the PIT distribution and a uniform distribution $U(0, 1)$, highlighting the relative rather than absolute numbers.
%\red{Can p-values be supplied for each statistic? The statistics themselves are difficult to interpret, other than ``lower is better'' (p-value in skgof was broken, having trouble finding 1-sample KS calculation for uniform distribution)}
\metaphor\ and \lephare\ perform well under the AD but poorly under the KS and CvM due to their high catastrophic outlier rates.
\annz\ and \flexzboost\  are the top scorers under these metrics of the PIT distribution.
\annz's strong performance can be attributed to an aspect of the training process in which training set galaxies with a PIT that more closely matches the percentiles of the DC1 training set's redshift distribution are upweighted; in effect, these quantile-based metrics were part of the algorithm itself that may or may not serve it well under more realistic experimental conditions.

\subsection{Performance on individual \pzpdf s}
\label{sec:cdelossresults}

The values of the CDE loss statistic of individual \pzpdf\ accuracy are provided in Table~\ref{tab:cdeloss}.
It is worth noting that strong performance on the CDE loss should imply strong performance on the other metrics, though the inverse is not necessarily true.
Thus the CDE loss is the most effective metric for generic science cases.

\begin{table}  %%% DATA TABLE %%%
\centering
\caption{CDE loss statistic of the individual \pzpdf s for each code.
A lower value of the CDE loss indicates more accurate individual \pzpdf s, with \cmnn\ and \flexzboost\ performing best under this metric.}
\label{tab:cdeloss}
\begin{tabular}{lr}
\hline
Photo-$z$ Code & CDE Loss \\
\hline
\annz 	    & $-6.88$ \\
\bpz 		    & $-7.82$ \\
%\textsc{Delight} 	& $-4.06$ \\
\delight    & $-8.33$\\
%\textsc{EAZY} 		& $-7.97$ \\
\eazy       & $-7.07$ \\
%\textsc{FlexZBoost} & $-11.51$ \\
\flexzboost & $-10.60$\\
\gpz		    & $-9.93$ \\
\lephare 	  & $-1.66$ \\
\metaphor 	& $-6.28$ \\
%\textsc{CMNN} 		& $-$ \\
\cmnn       & $-10.43$ \\
\skynet 	  & $-7.89$ \\
\tpz 		    & $-9.55$ \\
\hline
\trainz		  & $-0.83$ \\
%\textsc{Frankenz}	& $-$  \\
%\hline
\end{tabular}
\end{table}

This metric is the only one that can appropriately penalize \trainz\ and indicates strong performance for \cmnn\ and \flexzboost, the latter of which is optimized for this metric.
%Empirically, we have found that PIT RMSE is not as closely correlated to CDE loss as it is to the $N(z)$ statistics.

%\subsection{Response of Individual Codes}
%\label{sec:res:pz_indiv_codes}
%\red{this may be incorporated in to 5.1 and 5.2 rather than live in its own section!}


%\red{HOW CODES deal with negative fluxes, and magnitude uncertainties}

%\red{overall conclusions of the Results section}


\section{Discussion and future work}
\label{sec:discussion}
%(Sam Schmidt, Ofer Lahav, Jeff Newman, Alex Malz, Eve Kovacs, Tony Tyson, Tina Peters)

%The goal was not to determine a ``best'' photo-$z$ code.

% If methods can not reach the goals on idealized data, then they will almost surely not meet those same goals when the more complex problems that we expect to arise from real LSST data are included.
% The results presented in this paper enable an evaluation of which algorithms are the most promising moving forward, and potentially point to avenues for improvement.

In contrast with other \pzpdf\ comparison papers that have aimed to identify the ``best'' code for a given survey, we have focused on the somewhat more philosophical questions of how to assess \pzpdf\ methods and how to interpret differences between codes in terms of \pzpdf\ performance.


\subsection{Interpretation of metrics}
\label{sec:caution}
%Caution on metrics and photo-$z$ codes}\label{sec:caution}
%(Alex Malz, Sam Schmidt)

We remind the reader that contributed codes were given a goal of obtaining accurate \pzpdf s, not an accurate stacked estimator of the redshift distribution, so we do not expect the same codes to necessarily perform well for both classes of metrics.
Furthermore, our metrics are not necessarily able to assess the fidelity of individual \pzpdf s relative to true posteriors.
Metric-specific performance implies that we may need multiple \pzpdf\ approaches tuned to each metric in order to maximize returns over all science cases in large upcoming surveys.

The \trainz\ estimator, which assigns every galaxy a \pzpdf\ equal to $N(z)$ of the training set as described in Section~\ref{sec:trainz}, is introduced as a control case to demonstrate this point via \textit{reductio ad absurdum}.
Because our training set is perfectly representative of the test set, $N(z)$ should be identical for both sets down to statistical noise.
\textit{We make the alarming observation that \trainz\ outperforms all codes on the PIT-based metrics, and all but one code on the $N(z)$ based statistics.}
%\textit{Explain the connection between \trainz\ and these metrics.}

The CDE loss and point estimate metrics appropriately penalize \trainz's naivete.
As shown in Appendix~\ref{sec:pointmetrics}, \trainz ~has identical $ZPEAK$ and $ZWEIGHT$ values for every galaxy, and thus the \pz\ point estimats are constant as a function of true redshift, i.e. a horizontal line at the mode and mean of the training set distribution respectively.
The explicit dependence on the individual posteriors in the calculation of the CDE loss, described in Section~\ref{sec:cdelossresults}, distinguishes this metric from those of the \pzpdf\ ensemble and stacked estimator of the redshift distribution, despite their prevalence in the \pz\ literature.

% While looking at one, or even most of our metrics would have given the impression that this estimator was nearly optimal, red flags in CDE loss and point estimates reveal the problems.
%[more caution on over-reliance on metrics, and making sure that metrics test all of what you want] \red{this definitely still needs work}
In summary, context is crucial to defending against deceptively strong performers such as \trainz; the best \pzpdf\ method is the one that most effectively achieves our science goals, not the one that performs best on a metric that does not reflect those goals.
In the absence of clear goals or the information necessary for a principled metric definition, we must think carefully before choosing a single metric.

Trends in the metrics shared among codes indicate the

One obvious trend in several of the codes tested was an overall over or underprediction of the widths of \pzpdf s, as evidenced by the QQ plots and PIT histograms shown in Figure~\ref{fig:pitqq}.
A more careful tuning of bandwidth or smoothing during the validation process appears to be necessary for many of the machine learning based codes in order to improve the accuracy of $p(z)$.
For narrow peaked $p(z)$ the parameterization of the PDF as evaluated on a fixed redshift grid could also have contributed to some overestimates of $p(z)$ width simply due to the finite resolution.
After evaluating results such as those presented in \citet[]{Malz:qp}, in future analyses we plan to switch from a fixed grid to quantile-based storage of $p(z)$ in order to more efficiently and accurately store redshift PDF results.

Another important factor to keep in mind when examining the results presented in this paper is the fact that they are at some level dependent on the metrics that we aim to optimize: in this case code participants were asked to submit their optimal measures of an accurate $p(z)$, so participants used the training/validation data to optimize their codes accordingly.
Had we, instead, asked for an optimal $\hat{N(z)}$ the resulting metrics would be different for most, if not all, of the codes, as they would optimize toward a different goal.
Specific metric choice can affect which codes are among the ``best'' codes.
As stated earlier, there are cosmological science cases that require either individual galaxy photo-$z$ measures, or ensemble $\hat{N}(z)$ measures.
We must be aware of that the optimal method for one is not necessarily optimal for the other, and in fact several photo-$z$ algorithms may be necessary in the final cosmological analysis in order to satisfy the requirements of all science use cases.

The example of the simple \trainz\ estimator described in Section~\ref{sec:caution} shows a simple model with a $p(z)$ that is unrealistic for individual objects can still score very well on many of our metrics.
It is important to look at {\it all} metrics, and keep in mind what information each metric conveys.
%\red{mention 'null' and how 'dumb' algorithms can score well on metrics, need to keep all metrics in mind}
%\aim{[reiterate the meaning of $p(z)$ and the goal -- emphasize finding ``best'' code in terms of impact of assumptions underlying the method \textit{or} establishing methodology for the realistic case of biased prior information and other science goals]}
%\red{[the challenges faced in this project]}
%\red{mention z$<$2, missing some important degeneracies, new sim will go to higher z}
%\red{no quality cuts included, could identify outliers, but could also induce biases}
%\red{[how optimization defined has impact, here we optimized p(z), not n(z), could be science-case specific if stringent requirements are needed, though that may be a computational/storage challenge if too many use cases needed.]}
%\red{some discussion of output parameterization, limitations of 200 point fixed grid, in future switch to something like quantiles (cite qp paper)}

\subsubsection{Nontrivial experimental design}
\label{sec:futureexperiments}

%\begin{enumerate}
%\item \red{the combination of codes}

%\item \red{$p(z,\alpha)$}
%\item \red{tomographic analysis later on}
%\red{this paper deals with training, future papers will also explore calibration via cross-correlation methods}
%\end{enumerate}
%\red{mention the SRM once again, say that the plans are extensive, but we do have a plan and a rough timeline.}

The work presented in this paper is only the first step in characterizing current photo-$z$ codes and moving toward an improved photometric redshift estimator.
This initial paper explored code performance in idealized conditions with perfect catalog-based photometry and representative training data.
As mentioned in Section~\ref{sec:metrics} for the stacked $N(z)$ metrics we examined only the entire galaxy population with no selections in either photo-$z$ ``quality" or redshift.
The cosmological analyses for weak lensing and large scale structure based measures plan to break galaxy samples into tomographic redshift bins, using photo-$z$ $p(z)$ to infer the redshift distribution for each bin.
The specific selection used to determine these bins, both algorithmically and the specific bin boundaries, could induce biases due to indirect selections inherent in the photo-$z$ or other bin selection parameters.
The effects of tomographic bin selection will be explored in a dedicated future paper, including propagation of redshift uncertainties in a set of fiducial tomographic redshift bins in order to estimate impact on cosmological parameter estimation.

%\red{[are there any references for this?  I remember Gary Bernstein talking about this at a photo-z workshop in Japan, but I don't know that it was published.  I believe Michael Troxel has discussed this as well.]}
%\item \red{cosmological parameters in DC2}
%\item \red{mention z$<$2, missing some important degeneracies, new sim will go to higher z}
%\red{no quality cuts included, could identify outliers, but could also induce biases}

We re-emphasize that the dataset tested was quite idealized, and discuss enhancements that will be added in future simulations to test photo-$z$ codes on increasingly realistic conditions in the following section.

\subsubsection{More realistic mock data}
\label{sec:futuredata}

%\item \red{inclusion of incompleteness effects on training sample, ``wrong redshifts'', filter curves}
%\item \red{inclusion of photometric errors, image based effects, blending, lensing, sky, mask, etc...}

In future papers a focus of the \lsstdesc\ \Pz\ Working Group will be to add more and more complexity to our simulated data in order to test photo-$z$ algorithms in increasingly realistic conditions.
The most pressing concern is the impact of incomplete spectroscopic training samples.
The SEDs for the galaxy sample in this paper were constructed form linear combinations of five basis SED templates.
Future simulations will also include more complex SED information, with a more realistic range of physical properties, and the inclusion of AGN effects, a more insidious problem, where AGN features may not be apparent, but the colors and other host galaxy properties are perturbed relative to galaxies with an inactive nucleus.
In such cases, the presence of the AGN may induce a bias if the template SEDs or empirical datasets do not include low-level AGN counterparts.

As discussed extensively in \citet{Newman:2015} a representative set of spectroscopically confirmed galaxies spanning the full range of both redshift and apparent magnitude is necessary as a training set to characterize the mapping from broad-band fluxes to photometric redshifts.
%However, due to a combination of factors due to both the galaxy SEDs and limitations of spectrographic instruments, redshift samples are known to be systematically incomplete, where certain galaxy types and redshift intervals fail to yield a redshift even at the longest integration times on current and near-future instruments.  The more representative the training data, the better the performance of photo-$z$ algorithms will be.
Current and upcoming surveys are putting significant effort into obtaining these training samples \citep[e.~g.\,][]{Masters:2017}, however we still expect significant incompleteness for LSST-like samples, particularly at faint magnitudes.
We plan to produce a realistically incomplete training set of spectroscopic galaxies, modeling the performance of spectrographs, emission-line properties, and expected signal-to-noise to determine which galaxies will fail to yield a secure redshift.
In addition to outright redshift failures we will model the inclusion of a small number of falsely identified secure redshifts where misidentified emission lines or noise spikes cause an incorrect redshift solution to be marked as a high quality identification.
Even sub-per cent level contamination by false redshifts can impact \pz\ solutions at levels comparable to the stringent  requirements of some LSST science cases.
We expect different systematics to occur in different \pzpdf\ codes in response to training on incomplete data, particularly some of the machine learning methods.
The response of the codes will inform future directions of code development.

The underlying dataset limited this work to a maximum redshift of $z=2$.
LSST imaging after 10 years of observations will include a significant number of $z>2$ galaxies in expected cosmology samples, and their inclusion does have potential significant implications for photo-$z$ measures: the high redshift galaxies lie at fainter apparent magnitudes and can have anomalous colours due to evolution of stellar populations and the shift to rest-frame magnitudes probing UV features of the underlying SED.
More importantly, one of the most common ``catastrophic outlier'' degeneracies observed in deep photometric samples occurs when the Lyman break is mistaken for the Balmer break, leading to multiple redshift solutions at $z\sim0.2-0.3$ and $z\sim2-3$ \citep{Massarotti:2001}.
This degeneracy, along with other potential degeneracies, are currently not covered by the limited redshift range of this initial paper, which could mean that we are not probing the full range of potential extreme outlier populations and how our \pz\ estimators respond to them.
Extending simulations to include the high-redshift galaxy population will be a priority in future data challenges.

This initial paper explored a data set that was constructed at the catalog level, with no inclusion of the complications that come from measuring photometry from images.
Future data challenges will move to catalogs constructed from mock images, including effects that will have great impact on \pz\ estimates, which will naturally include the complications of object blending, sensor effects, different observing conditions, amongst others.
Object blending will be a major area of investigation, as the mixing of flux from multiple objects and the resultant change in measured colours is predicted to affect a large fraction of LSST galaxies \citep{Dawson:2016}, and will be one of the major contributing systematics for photo-$z$'s.

In this study we have not accounted for the presence of Active Galactic Nuclei (AGN) contributions to galaxy fluxes.
In some cases, AGN will be easily identified from the colors and morphologies, i.e. the case of the brightest quasars where the nuclear activity outshines the host galaxy, and numerous studies have utilized color selection to create large samples of quasars \citep[e.g.][]{Richards:06,Maddox:08,Richards:15}.
In current deep fields, similar in depth to what we expect from \lsst, variability information and multi-wavelength data have been critical to not only identify AGN dominated galaxies, but also obtain more accurate photometric redshifts \citep[e.g][]{Salvato:11}.

In addition to AGN dominated galaxies, those with lower levels of nuclear activity present a more insidious problem, where AGN features may not be apparent, but the colours and other host galaxy properties are perturbed relative to galaxies with an inactive nucleus.
In such cases, the presence of the AGN may induce a bias if the template SEDs or empirical datasets do not include low-level AGN counterparts.
For LSST, we will need to identify and obtain accurate photometric redshifts of all types of AGN for a range of science goals, whether it is to eliminate such objects from cosmology experiments, or to use them with confidence, all the way through to understanding galaxy evolution and the role that AGN may play in influencing galaxy properties over cosmic time.

A promising route to classifying and obtaining accurate photometric redshifts for the AGN population is by combining machine learning with template-fitting techniques, as has recently been demonstrated by \citet{Duncan:18} for radio-selected AGN.
This is because AGN are relatively easy to obtain spectroscopic redshifts for over all redshifts due to the strong emission lines that they exhibit, allowing very good training sets for machine learning algorithms to use.
Whereas for those galaxies where the AGN is sub-dominant the galaxy templates are still adequate for obtaining reasonable photometric redshifts.

In addition to these improvements, the \lsstdesc \Pz\ Working Group plans to look at all potential methods to combine the results from multiple \pzpdf\ codes to improve accuracy, similar to the work presented in \citet{Dahlen:13,Carrascokind:14,Duncan:18}.
Taking advantage of multiple algorithms that use observables in slightly different ways has shown promise, however we must be very conscious of whether a potential combination properly treats the covariance between the methods, given that they are estimating quantities based on the same underlying observables.
Several science cases wish to estimate physical quantities along with redshift, for example galaxy stellar mass and star formation rate.
Proper joint estimation of redshift and physical quantities requires an in depth understanding of galaxy evolution, and progress on accurate bivariate redshift probability distributions will go hand in had with progress on understanding galaxies themselves.
Parameterization and storage of a complex 2-dimensional probability surface for potentially billions of galaxies (or even subsets of hundreds of thousand of particular interest) pose a potential challenge.
These issues will be examined in another future paper.

Finally, while this paper and future papers discussed above focus on photometric redshift codes and estimating accurate \pzpdf s from training data, we plan a separate, but complementary, project to examine calibration of the resultant redshifts via spatial cross-correlations \citep{Newman:2008}, which will be explored in a separate series of future papers.
The overarching plan describing everything laid out in this section is described in more detail in the \lsstdesc\ Science Roadmap (see Footnote in Section~\ref{sec:intro}).
These plans will require significant effort, but they are necessary if we are to make optimal use of the LSST data for astrophysical and cosmological analyses.


%\section{Conclusion}
\label{sec:conclusion}
%(Sam Schmidt, Ofer Lahav, Jeff Newman, Alex Malz, Eve Kovacs, Tony Tyson, Tina Peters)

This paper compares the performance of twelve \pzpdf\ approaches under idealized experimental conditions of representative and complete prior information to set a baseline for an upcoming sensitivity analysis by first isolating the impact of the \pzpdf\ estimation technique from realistic physical systematics of the data.
Though the mock data set of this experiment did not include true \pz\ posteriors for comparison, \textbf{we interpret deviations from perfect results given perfect prior information as the imprint of the implicit assumptions underlying the estimation approach}.

We evaluated the codes under science-agnostic metrics both established and emerging to stress-test the ensemble properties of \pzpdf\ catalogues derived by each method, as well as metrics of a prevalent summary statistic of \pzpdf\ catalogues used in cosmological analyses.
We observe that no one code dominates in all metrics, and that the standard metrics of \pzpdf s and the stacked estimator of the redshift distribution can be gamed by a trivial approach to \pzpdf\ estimation.
We emphasize to the \pz\ community that \textbf{metrics used to vet \pzpdf\ methods must be tailored to science cases with self-consistent analysis approaches that respect the inherently probabilistic nature of \pzpdf\ catalogs}.


% ----------------------------------------------------------------------

\subsection*{Acknowledgments}

%%% Here is where you should add your specific acknowledgments, remembering that some standard thanks will be added via the \code{desc-tex/ack/*.tex} and \code{contributions.tex} files.

%This paper has undergone internal review in the LSST Dark Energy Science Collaboration. % REQUIRED if true

Author contributions are listed below. \\
S.J.~Schmidt: Co-led the project. (conceptualization, data curation, formal analysis, investigation, methodology, project administration, resources, software, supervision, visualization, writing -- original draft, writing -- review \& editing). \\
A.I.~Malz: Co-led the project, contributed to choice of metrics, implementation in code, and writing. (conceptualization, methodology, project administration, resources, software, visualization, writing -- original draft, writing -- review \& editing). \\
J.Y.H.~Soo: Ran ANNz2 and Delight, updated abstract, edited sections 1 through 6, added tables in Methods and Results, updated references.bib and added references throughout the paper. \\
I.A.~Almosallam: vetted the early versions of the data set and ran many photo-z codes on it, applied GPz to the final version and wrote the GPz subsection. \\
M.~Brescia: Main-ideator of MLPQNA and co-ideator of METAPHOR; modification of METAPHOR pipeline to fit the LSST data structure and requirements. \\
S.~Cavuoti: Co-ideator of METAPHOR, contributed to choice and test of metrics, ran METAPHOR, minor text editing. \\
J.~Cohen-Tanugi: contributed to running code, analysis discussion, and editing, reviewing the paper. \\
A.J.~Connolly: Developed the colour-matched nearest-neighbours photo-z code; participated in discussions of the analysis. \\
J.~DeRose: One of the primary developers of the Buzzard-highres simulated galaxy catalogue employed in the analysis. \\
P.E.~Freeman: Contributed to choice of CDE metrics and to implementation of FlexZBoost. \\
M.L.~Graham: Ran the colour-matched nearest-neighbours photo-z code on the Buzzard catalogue and wrote the relevant piece of Section 2; participated in discussions of the analysis. \\
K.G.~Iyer: assisted in writing metric functions used to evaluate codes. \\
M.J.~Jarvis: Contributed text on AGN to Discussion section and portions of GPz work. \\
J.B.~Kalmbach: Worked on preparing the figures for the paper. \\
E.~Kovacs: Ran simulations, discussed data format and properties for SEDs, dust, and ELG corrrections. \\
A.B.~Lee: Co-developed FlexZBoost and the CDE loss statistic, wrote text on the work, and supervised the development of FlexZBoost software packages. \\
G.~Longo: Scientific advise, test and validation of the modified METAPHOR pipeline, text of the METAPHOR section. \\
C.B.~Morrison: Managerial support; Discussions with authors regarding metrics and style; Some coding contribution to metric computation. \\
J.A.~Newman: Contributions to overall strategy, design of metrics, and supervision of work done by Rongpu Zhou. \\
E.~Nourbakhsh: Ran and optimized TPZ code and wrote a subsection of Section 2 for TPZ. \\
E.~Nuss: contributed to running code, analysis discussion, and editing,reviewing the paper. \\
T.~Pospisil: Co-developed FlexZBoost software and CDE loss calculation code. \\
H.~Tranin: contributed to providing SkyNet results and writing the relevant section. \\
R.H.~Wechsler: Project lead for Buzzard-highres simulated galaxy catalogue employed in analysis. \\
R.~Zhou: Optimized and ran EAZY and contributed to the draft. \\
R.~Izbicki: Co-developed FlexZBoost and the CDE loss statistic, and wrote software for FlexZBoost \\
 % Standard papers only: author contribution statements. For examples, see http://blogs.nature.com/nautilus/2007/11/post_12.html

% This work used TBD kindly provided by Not-A-DESC Member and benefitted from comments by Another Non-DESC person.

% Standard papers only: A.B.C. acknowledges support from grant 1234 from ...

The authors would like to thank their LSST-DESC publication review committee for comments that improved the paper draft.

{ \red{personal funding sources}}
S.~Schmidt acknowledges support from DOE grant DE-SC0009999 and NSF/AURA grant N56981C.
AIM is advised by David W. Hogg and was supported by National Science Foundation grant AST-1517237.
AIM was also supported by the U.S. Department of Energy, Office of 
Science, Office of Workforce Development for Teachers and Scientists, Office of 
Science Graduate Student Research (SCGSR) program, administered by the Oak 
Ridge Institute for Science and Education for the DOE under contract number 
DE‐SC0014664.

In addition to packages cited in the text, analyses performed in this paper used the following software packages: \textsc{Numpy} and \textsc{Scipy} \citep{numpyscipy}, \textsc{Matplotlib} \citep{matplotlib}, \textsc{Seaborn} \citep{seaborn}, \textsc{minFunc} \citep{minfunc}, \textsc{pySkyNet} \citep{pyskynet}, and {\textsc photUtils} from the LSST simulations package \citep{lsstphotutils}.

\input{desc-tex/ack/standard} % also available: key standard_short

% This work used some telescope which is operated/funded by some agency or consortium or foundation ...

% The DESC acknowledges ongoing support from the Institut National de Physique Nucleaire et de Physique des Particules in France; the Science \& Technology Facilities Council in the United Kingdom; and the Department of Energy, the National Science Foundation, and the LSST Corporation in the United States.
%
% DESC uses resources of the IN2P3 Computing Center (CC-IN2P3--Lyon/Villeurbanne - France) funded by the Centre National de la Recherche Scientifique; the National Energy Research Scientific Computing Center, a DOE Office of Science User Facility supported by the Office of Science of the U.S. Department of Energy under Contract No. DE-AC02-05CH11231; STFC DiRAC HPC Facilities, funded by UK BIS National E-infrastructure capital grants; and the UK particle physics grid, supported by the GridPP Collaboration.
%
% This work was performed in part under DOE Contract DE-AC02-76SF00515.

% We acknowledge the use of An-External-Tool-like-NED-or-ADS.

%{\it Facilities:} \facility{LSST}

\appendix

\section{Point Estimate Photometric Redshifts}
\label{sec:pointmetrics}
While we do not recommend the use of single point estimates of redshift for most science applications, plots of the point estimates can be a useful qualitative diagnostic of photo-z code performance, i.~e.~examining point photo-$z$ vs.~spec-$z$ plots visually can give a quick impression of some common trends in different codes.  Computing point estimate statistics may also be useful for more direct comparisons with previous photo-z evaluations.  If a point-estimate is preferred for a specific science case, it is fairly simple to compute the mean, mode, or some other simple estimator from each $p(z)$, so these point estimates can be easily derived from the stored $p(z)$.

There are several common point estimators of photo-$z$ posteriors employed by different codes, e.~g. the mode, mean, median of the $p(z)$ distribution.  In addition, many of the machine learning based estimators can be set up to return a single redshift solution.  For example, SkyNet can be configured to run as a regressor that returns a single float rather than a classifier that returns a 200-bin $p(z)$ estimate.  The single value returned by a machine learning based code may not correspond to a particular measure such as the mode or mean, and so to avoid interpretation of results that might be introduced by variations in choice of specific point-estimate implementation per code, we discard the code-specific point estimates. We instead calculate point estimates more uniformly across the codes directly from the $p(z)$ using two measures, $z_{PEAK}$ and $z_{WEIGHT}$.  $z_{PEAK}$  is simply the maximum value attained for each galaxy p(z), the mode of the probability distribution.  $z_{WEIGHT}$ is defined similarly to how it is defined in \citet{Dahlen:13}, as the weighted mean of the redshift over the {\it main peak} of $p(z)$ containing the $z_{PEAK}$ value.  The main peak is defined by subtracting 0.05$\,\times\,z_{PEAK}$ from $p(z)$ and identifying the roots to isolate the peak containing $z_{PEAK}$, $z_{WEIGHT}$ is defined as the weighted mean redshift within this peak.  We restrict to a single peak in order to avoid confusion from bimodal and multimodal $p(z)$ such as those shown in bottom panels of Figure~\ref{fig:pz_examples}.  For example, for a bimodal probability distribution a weighted mean calculated over both peaks would fall between the peaks, at a redshift where the probability is minimal. Restricting the weighting to a single peak ensures that the point estimate will fall in the region of maximum redshift probability. %\red{Have Dritan check zweight description to make sure this is what was done.}
% The most common point estimators of photo-$z$ PDFs are the mode, median, and mean of the distribution.  Following \cite{Tanaka:17}, we report the values of the following metrics using whichever of these point estimators performs best for each method, noting which point reduction is used in each case.

\subsection{Point Estimate Metrics}
\label{sec:point_metrics}
We calculate the commonly used point estimate metrics of the overall photo-$z$ scatter ($\sigma_{z}$, the standard deviation of the photo-$z$ residuals), bias, and ``catastrophic outlier rate''.  Specifically, we calculate the metrics as follows:
we define $e_{z}$ as

\begin{equation}
e_{z} = \frac{z_{P} -z_{S}}{1+z_{S}}
\end{equation}
where $z_{P}$ is the point estimate and $z_{S}$ is the true redshift.
In practice, because the standard deviation calculation is quite sensitive to the outliers, we define the photo-$z$ scatter, $\sigma$ in terms of the Interquartile Range (IQR), the difference between the 75th and 25th percentiles of the $e_{z}$ distribution.  In order to match the usual meaning of a 1$\sigma$ interval, we scale the IQR and define $\sigma_{IQR} = IQR/1.349$, as there is a factor of 1.349 difference between the IQR and the standard deviation of a Normal distribution.
%$\sigma_{z}$ is, in practice, defined in terms the Interquartile Range (IQR) of $e_{z}$ values as $IQR/1.349$, where $IQR$ is the difference between the 75th and 25th percentiles of the $e_{z}$ distribution.  The factor of 1.349 relates the IQR of a Normal distribution to the standard 68\% range of a one $\sigma$ uncertainty.
%\scc{usually the statistical indicator defined as $\sigma$ is the standard deviation I know that is not a rule but if someone skips the test and go directly to the table (and we all know that a lot of people skips the definition of statistical indicator) could be confusing, could we report it as $\frac{IQR(e_z)}{1.349}$ or $\sigma_{IQR}$ that should be clear without any risk of confusion?}\red{Using IQR is done because IQR is less sensitive to outliers than calculating a sigma.  I've rewritten the text explaining this above.--SJS}
While many other studies define the bias based on the {\it mean} offset between true and estimated redshift, in this study we define the bias as the median value of $e_{z}$ for the sample.  We use median as it is, once again, less sensitive to outliers than the mean.  The catastrophic outlier fraction is defined as the fraction of galaxies with $e_{z}$ greater than the {\it larger} of $3\sigma_{IQR}$ or 0.06, i.e. 3$\sigma$ outliers with a floor of $\sigma_{IQR}$=0.02.
%\scc{again, is clearly not a rule but a huge number of paper refers to the mean as bias, could we use the word median, which is clear without any doubt rather than bias?}
%\red{I also have code to calculate $\sigma_{MAD}$, should we include this as well?  It's almost always within a few thousandths of $\sigma_{IQR}$, so I left it out for now}.
%\scc{we may just say that the two indicators reports almost the same value therefore there is no need for both of them?}
For reference, the goals stated in Section 3.8 of the LSST Science Book \citep{Abell:09} for photo-$z$ performance in these metrics, assuming perfect training knowledge (as we are testing in this paper) are:
\begin{itemize}
\item RMS scatter$ < 0.02(1+z)$
%\scc{actually the SB reports in page 75 that the goal of 0.02 is for the RMS of $\frac{\sigma_{z}}{(1+z)}$ while we are using IQR}
\item bias $<$ 0.003 %\scc{in page 518 of the SB is clearly reported that the mean is expected to be less than 0.003 in our case we have defined as bias the median and not the mean }
\item catastrophic outlier rate $<$ 10\% %\scc{clearly the number of outliers since are defined above $3\sigma$ strictly depends on the sigma definition, in the science book it seems to refer to the RMS of the $e_z$ distribution even ef it could be the standard deviation, but again I don't think it refers to IQR, therefore we could not consider exactly this numbers as reference.}
%\red{I am going off of conversations with Zeljko and how we {\it actually} computed the metrics in the Science Book (I have the scripts).  You are correct that several of these are slightly different than actually stated in the Science Book, or where the Science Book does not use very precise language as to what was done to compute the RMS or define the bias.  We can also cut out those specifics, or maybe say Ivezic private communication, maybe?  Also $\sigma_{z}$ defined in terms of IQR is exactly equivalent to the standard deviation when scaled by 1.349 for a Gaussian distribution, so the numbers are appropriate to cite in this context.  IQR is just a more robust way of calculating sigma for a distribution with outliers.--SJS }
%\scc{I am very sorry I am not saying that IQR is not fine, I want to keep the IQR my concerns are related to the notation and the \textit{reader understanding}, if I wrote $\sigma$ somewhere in a table the reader will understand for sure that is the traditional standard deviation (a lot of people skips the indicators definition for the \textit{obvious} ones), I am suggesting to replace the notation $\sigma$ with the notation $\sigma_{IQR}$ (or something like this) to avoid any confusion, and also to change from $bias$ to $median$ for the same reason. I agree that say Ivezic private communication could be better.}
\end{itemize}
These definitions are similar, but not exactly the same, as the $\sigma_{IQR}$ and median bias calculated here, but are similar enough for qualitative comparisons to the LSST goals.

%Scatter: SRD $\sigma<0.02(1+z)$
%Catastrophic Outler Rate: SRD $3\sigma$ ``catastrophic outlier'' rate below 10 per cent
%Bias: SRD bias$<0.003$

Fig.~\ref{fig:pz_pointestimates}  shows the point estimates for both $z_{PEAK}$ and $z_{WEIGHT}$.  Point density is shown with mixed contours to emphasize that most of the galaxies do fall close to the $z_{phot}=z_{spec}$ line, while blue points show differing characteristics of the outlier populations.  The red dashed lines indicated the cutoff for catastrophic outliers, defined as: $max(0.06,3\sigma_{IQR})$.  As with the full $p(z)$ results, a variety of behaviours are evident in the different codes.  Table~\ref{tab:pointestimates} lists the scatter, bias, and catastrophic outlier fractions for the codes.  The performance of the codes for point metrics is highly correlated with performance on $p(z)$ based tests, which is to be expected, given that the point-estimates were derived from the $p(z)$.  Some discretization is evident in $z_{PEAK}$, particularly for \textsc{SkyNet}, due to the finite grid spacing of the reported $p(z)$.  These discreteness effects are mitigated by the weighting of $z_{WEIGHT}$, resulting in a smoother distribution of redshift estimates.  Several features perpendicular to the main $z_{phot}=z_{spec}$ line are evident.  These features are due to the 4000 angstrom break passing through the gaps between adjacent LSST filters.  These features are most prominent in template-based codes, but appear to some degree in all codes tested.

In even the best performing codes, there are visible occupied regions away from the $z_{phot}=z_{spec}$ line, corresponding to degenerate redshift solutions for certain LSST magnitudes and colors.  While use of the full information available via $p(z)$ mitigates their impact, a full understanding of the outlier population is critical for LSST science, particularly in tomographic applications %\red{I forget what exactly I was trying to say here, if someone else wants to take a crack at this it might be helpful --SJS}
%\scc{may be worth to say that the usage of further bands could mitigate this effect by removing the degeneration  usually induced from emission lines moving across different filters?}

Finally, we note that all eleven codes are at or near the goals for point-estimates as outlined in the LSST Science Requirements Document\footnote{available at: \url{http://ls.st/srd}} and \citet{Graham:17}.  This is to be expected, given that the requirements were designed such that a point estimate photo-z would meet these requirements for perfect training data to a depth of $i<25$.  But, it is still an encouraging sign, given an updated mock galaxy simulation and the expanded set of photo-$z$ codes tested.
%\red{maybe say consistent with expectations, and a reference to Melissa's paper in here?  What else to say about nearly meeting point goals?}

\begin{figure*}
\centering
\includegraphics[width=0.9\textwidth]{fig/ZPEAK_szpz_threecolumn_12codes_navy.jpg}\\
\includegraphics[width=0.9\textwidth]{fig/ZWEIGHT_szpz_threecolumn_12codes_navy.jpg}
\caption{Point estimate photo-z's derived from the posteriors. Top panel shows $z_{PEAK}$, while bottom panel shows $z_{WEIGHT}$.  Point estimate density is represented with fixed density contours, while outliers at lower density are represented by blue points.  While use of point-estimate photo-$z$'s is not recommended, they do make for useful comparative and visual diagnostics.  In the lower-right panel of each plot, the \trainz\ estimator results in identical photo-$z$ estimates at the mode and mean of the training set $N'(z)$ distribution for all galaxies.} \label{fig:pz_pointestimates}
\end{figure*}


\begin{table*}
\begin{center}
\caption{Point estimate statistics}\label{tab:pointestimates}
\begin{tabular}{lcccccc}
\hline
\hline
                 &            & $Z_{PEAK}$  &          &  & $Z_{WEIGHT}$          &\\
\hline
Photo-z Code       & $\frac{\sigma_{IQR}}{(1+z)}$ & median  & \multicolumn{1}{|p{0.75cm}|}{\centering outlier \\fraction} & $\frac{\sigma_{IQR}}{(1+z)}$ & median & \multicolumn{1}{|p{0.75cm}|}{\centering outlier \\fraction}\\
\hline
\textsc{ANNz2}     & 0.0270  &  0.00063  & 0.044      & 0.0244  &  0.000307  & 0.047  \\
\textsc{BPZ}       & 0.0215  & -0.00175  & 0.035      & 0.0215  & -0.002005  & 0.032 \\
\textsc{Delight}   & 0.0212  & -0.00185  & 0.038      & 0.0216  & -0.002158  & 0.038 \\
\textsc{EAZY}      & 0.0225  & -0.00218  & 0.034      & 0.0226  & -0.003765  & 0.029 \\
\textsc{FlexZBoost}& 0.0154  & -0.00027  & 0.020      & 0.0148  & -0.000211  & 0.017 \\
\textsc{GPz}       & 0.0202  & -0.00091  & 0.036      & 0.0201  & -0.000950  & 0.037 \\
\textsc{LePhare}   & 0.0236  & -0.00161  & 0.058      & 0.0239  & -0.002007  & 0.056 \\
\textsc{METAPhoR}  & 0.0264  &  0.00000  & 0.037      & 0.0262  &  0.001333  & 0.048 \\
\textsc{CMNN}        & 0.0184  & -0.00132  & 0.035      & 0.0170  & -0.001049  & 0.034 \\
\textsc{Skynet}    & 0.0219  & -0.00167  & 0.036      & 0.0218  &  0.000174  & 0.037 \\
\textsc{TPZ}       & 0.0161  &  0.00309  & 0.033      & 0.0166  &  0.003048  & 0.031 \\
\hline
\trainz	   & 0.1808  &  -0.2086  & 0.000	  & 0.2335  & 0.022135  & 0.000\\
\end{tabular}
\end{center}
\end{table*}


% Include both collaboration papers and external citations:
\bibliography{main,lsstdesc}

\end{document}

% ======================================================================
