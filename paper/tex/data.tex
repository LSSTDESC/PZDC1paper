\section{Data}
\label{sec:sims}

In order to test the current generation of \pzpdf\ codes, we employ an existing simulated galaxy catalogue, described in detail in Section~\ref{sec:buzzard}.
The experimental conditions shared among all codes are motivated by the \lsst\ SRD requirements and implemented for machine learning and template-based \pzpdf\ codes according to the procedures of Sections~\ref{sec:buzztraining} and \ref{sec:buzztemplates} respectively.
Finally, in Section~\ref{sec:limitations}, we elaborate on the scope of the experimental conditions considered as a result of the sophistication of the mock data.

\subsection{Buzzard-v1.0 simulation}
\label{sec:buzzard}

Our mock catalogue is derived from the \textsc{Buzzard-highres-v1.0}  of De Rose et al., in prep, Wechsler et al., in prep) catalogue.
% UPDATE THIS CITATION NOW THAT THE PAPER IS OUT
\textsc{Buzzard} is built on a dark matter-only N-body simulation of $2048^{3}$ particles in a $400$ Mpc h$^{-1}$ box.
The lightcone was constructed from smoothing and interpolation between a set of time snapshots.
Dark matter halos were identified using the \textsc{Rockstar} software package \citep{Behroozi:13} and then populated with galaxies with a stellar mass and absolute $r$-band magnitude in the \sdss\ system determined using a sub-halo abundance matching model constrained to match both projected two-point galaxy clustering statistics and an observed conditional stellar mass function \citep{Reddick:13}.

To assign an SED to each galaxy, the {\it Adding Density Dependent Spectral Energy Distributions} (\texttt{ADDSEDS}, deRose in prep.)\footnote{\url{https://github.com/vipasu/addseds}} procedure was used.
\texttt{ADDSEDS} uses a sample of $\sim 5\times 10^{5}$ galaxies from the magnitude-limited \sdss\ Data Release 6 Value Added Galaxy Catalogue \citep{Blanton:05} to train an empirical relation between absolute $r$-band magnitude, local galaxy density, and SED.
Each \sdss\ spectrum is parameterized by five weights corresponding to a weighted sum of five basis SED components using the \texttt{k-correct \red{v4\_3?}} software package\footnote{\url{http://kcorrect.org}} \citep{Blanton:07}.

Correlations between SED and galaxy environment were included so as to preserve the colour-density relation of galaxy environment.
The distance to the spatially projected fifth-nearest neighbour was used as a proxy for local density in the \sdss\ training sample.
For each simulated galaxy, a galaxy with similar absolute $r$-band magnitude and local galaxy density was chosen from the training set, and that training galaxy's SED was assigned to the simulated galaxy.

Given the SED, absolute $r$-band magnitude, and redshift, we computed apparent magnitudes in the six \lsst\ filter passbands, $ugrizy$.
We assigned magnitude errors in the six bands using the simple model of \citet{Ivezic:08}, assuming achievement of the full 10-year depth, with a modification of fiducial \lsst\ total numbers of 30-second visits for photometric error generation: we assume 60 visits in $u$-band, 80 visits in $g$-band, 180 visits in $r$-band, 180 visits in $i$-band, 160 visits in $z$-band, and 160 visits in $y$-band.
%numbers taken from Alex Abate's photErrorModel.py script in PhotoZDC1 repository, which was used with nYrObs=10.

As a consequence of adding Gaussian-distributed photometric errors, 2.0\% of our galaxies exhibit a negative flux in one or more bands, the vast majority of which are in the $u$-band.
We deem such negative fluxes \textit{non-detections} and assign a placholder magnitude of 99.0 in the catalogue to indicate to the \pzpdf\ codes that such galaxies would be ``looked at but not seen'' in multi-band forced photometry.

\subsection{Experimental Conditions}
\label{sec:controlled}

% NOW MOTIVATE THE SHARED PRIOR INFORMATION

\subsubsection{Training set selection}
\label{sec:buzztraining}

The total initial catalogue covered $400$ square degrees and contained $238$ million galaxies to an apparent magnitude limit of $r\!=\!29$ and spanning the redshift range $0\!<\!z\!\leq\!8.7$.  In order for statistical errors not to dominate, we need less than one million galaxies in our sample.
Several studies claim that only a few tens of thousands of spectra are necessary to calibrate photo-z surveys to Stage IV requirements (e.~g.~\citet{Bernstein:10}, \citet{Masters:2017}).
Therefore, we aim for a final number of training galaxies between $3\times 10^{4}$ and $5\times 10^{4}$ in our sample.
%This catalogue contained two orders of magnitude more galaxies than were necessary to determine statistics for this study,
In order to reduce our sample to a reasonable size, we limit our dataset to a subset of $\sim\!16.8$ square degrees selected from five separate spatial regions of the simulation.
Systematic problems with galaxy colors above $z\!>\!2$ were observed, so the catalogue was limited to include only galaxies in the redshift range $0\!<\!z\!\leq\!2.0$.
A random subset of the the remaining galaxies was chosen, and placed at random into either a ``training'' set ($10$ per cent of the sample), for which the galaxies true redshifts will be supplied, or a ``test'' set (the remaining $90$ per cent of the sample), for which each code will need to predict a redshift PDF for each galaxy.
%The resulting catalogues contain $111\,171$ training galaxies and $1\,000\,883$ test galaxies.
Finally, we restrict our analysis to a sample with an apparent i-band magnitude limit $i<25.3$, which give a signal-to-noise $\sim\,30$ for most galaxies, a cut often referred to as the expected ``LSST Gold Sample''.
This magnitude cut results in a training set with $44\,404$ galaxies and a test set containing $399\,356$ galaxies.  All subsequent results will evaluate this ``gold sample'' test set.
In order to blind results, initially redshifts were not revealed for the ``test'' set, and were only supplied for the training sample galaxies.
This prevented code runners from tweaking results and fitting to the specific test set.

\subsubsection{Template library construction}
\label{sec:buzztemplates}

As mentioned in Section~\ref{sec:buzzard}, the SEDs in the Buzzard simulation are drawn from an empirical set of SEDs taken from the SDSS DR6 NYU-VAGC, a sample of roughly $\sim5\times 10^{5}$ galaxies with spectra in SDSS.
To determine a finite set of templates to use with template fitting codes we take the five SED weight coefficients for each of the galaxies in the SDSS sample and run a simple K-means clustering algorithm on this five dimensional space.
Each dimension was normalized such that it spanned an interval $[0,1]$.
The K-means clusters partition the five-dimensional space of coefficients into Voronoi cells, spanning the space of coefficients in a way that properly reflects the underlying density in the coefficients.
Thus, the resultant SEDs constructed using the cell centers as weight coefficients will provide a reasonable spanning SED set.
An ad-hoc number of $K=100$ was chosen and the $100$ K-means centre positions are taken as the weights for the \textsc{k-correct} SED components to construct one hundred template SEDs. These $100$ templates were provided,
%\red{(JS: if I remembered correctly, weren't there 150 templates given? Answer: no, initially I supplied a list of 150 that were drawn with a different algorithm, but once we switched to the final k-means set I only supplied 100 (the initial algorithm over-weighted outlier SEDs, which is why we needed the extra 50)--SJS)}
and the templates were used by both \textsc{BPZ} and \textsc{LePhare}; however, because EAZY was designed and written to use the same five basis templates employed by \textsc{k-correct} when constructing our mock galaxies, EAZY was run using linear combinations of these five templates rather than using the 100 discrete templates.
The ability to fit for linear combinations of templates highlights an important implementation difference between similar photo-z codes.

\subsection{Limitations}
\label{sec:buzzlimitations}

For our initial investigation of photometric redshift codes, we begin with a data set that is somewhat idealized, and does not contain all of the complicating factors present in real data.
In several cases, the simplification is done with a purpose, with potentially confounding effects excluded in order to better isolate the differences between current-generation photo-$z$ codes and their causes.
We list several of the simulation limitations in this section.
As the simulation is catalogue-based, no image level effects, such as photometric measurement effects, object blending, contamination from sky background (Zodiacal light, scattered light, etc...), lensing magnification, or Galactic reddening are included.  No stars are included in the catalogue, nor are the effects of AGN.
As all SEDs are constructed from only five basis templates, properties of the galaxy population will be restricted to follow linear combinations of the characteristics of the five basis templates, so certain non-linear features, for example the full range of emission line fluxes relative to the continuum, will not be included in the model galaxy population.
Moreover, the linear combinations of templates are modeled on the $\sim5\times 10^{5}$ SDSS galaxies discussed in Section~\ref{sec:buzzard}, and thus only galaxies that resemble those spectroscopically observed by the SDSS will be included in the sample.
No additional dust reddening intrinsic to the host galaxy is included, the only approximation of dust extinction comes in the form of dust encoded in the five basis SEDs via the training set used to create the basis templates.
Simple linear combinations of these basis templates will, once again, not explore the full range of realistic dust extinction observed in galaxy populations.
While these idealized conditions limit the realism of our galaxy population, some are also by design.
We aim to test the photo-z codes at a very basic level, and a simplified model assures that differences in results seen between the codes are due to fundamental differences in their underlying assumptions and implementation details, rather than more nuanced properties.
