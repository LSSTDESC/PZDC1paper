\section{Data}
\label{sec:sims}

In order to test the current generation of \pzpdf\ codes, we employ an existing simulated galaxy catalogue, described in detail in Section~\ref{sec:buzzard}.
The experimental conditions shared among all codes are motivated by the \lsst\ SRD requirements and implemented for machine learning and template-based \pzpdf\ codes according to the procedures of Sections~\ref{sec:buzztraining} and \ref{sec:buzztemplates} respectively.

\subsection{The \textsc{Buzzard-v1.0} simulation}
\label{sec:buzzard}

Our mock catalogue is derived from the \textsc{Buzzard}-highres-v1.0 catalogue (DeRose et al., in prep).
\textsc{Buzzard} is built on the \texttt{Chinchilla-400} \citep{Mao:15} dark matter-only N-body simulation consisting of $2048^{3}$ particles in a $400$ Mpc h$^{-1}$ box.
The lightcone was constructed from smoothing and interpolation between a set of time snapshots.
Dark matter halos were identified using the \texttt{Rockstar} software package \citep{Behroozi:13} and then populated with galaxies with stellar masses and absolute $r$-band magnitudes in the \sdss\ system determined using a sub-halo abundance matching model constrained to match both projected two-point galaxy clustering statistics and an observed conditional stellar mass function \citep{Reddick:13}.

To assign a spectrum to each galaxy, the Adding Density Dependent Spectral Energy Distributions (SEDs) procedure \citep[\texttt{ADDSEDS,}][Wechsler et al., in prep,]{DeRose:19} was used.
\texttt{ADDSEDS} uses a sample of $\sim 5\times 10^{5}$ galaxies from the magnitude-limited \sdss\ Data Release 6 Value Added Galaxy Catalogue \citep[NYU-VAGC,][]{Blanton:05} to train an empirical relation between absolute $r$-band magnitude, local galaxy density, and SED.
Each \sdss\ spectrum is parameterized by five weights corresponding to a weighted sum of five basis SED components using the \texttt{k-correct} software package\footnote{\url{http://kcorrect.org}} \citep{Blanton:07}.

Correlations between SED and galaxy environment were included so as to preserve the colour-density relation of galaxy environments.
The distance to the spatially projected fifth-nearest neighbour was used as a proxy for local density in the \sdss\ training sample.
For each simulated galaxy, a galaxy with similar absolute $r$-band magnitude and local galaxy density was chosen from the training set, and that training galaxy's SED was assigned to the simulated galaxy.

\subsubsection{Caveats}
\label{sec:buzzlimitations}

By necessity, \textsc{Buzzard} does not contain all of the complicating factors present in real data, and here we discuss the most pertinent ways that these limitations affect our experiment.
\textsc{Buzzard} includes only galaxies, not stars nor AGN.
The catalogue-based construction excludes image-level effects, such as deblending errors, photometric measurement issues, contamination from sky background (Zodiacal light, scattered light, etc.), lensing magnification, and Galactic reddening.

The \textsc{Buzzard} SEDs are drawn from a set of $\sim 5 \times 10^{5}$ SEDs, which themselves are derived from a five-component linear combination fit to $\sim 5 \times 10^{5}$ \sdss\ galaxies; thus the sample contains only galaxies that resemble linear combinations of those for which \sdss\ obtained spectra, and there are necessarily duplicates.
The linear combination SEDs also restrict the properties of the galaxy population to linear combinations of the properties corresponding to five basis templates, precluding the modeling of non-linear features such as the full range of emission line fluxes relative to the continuum.
The only form of intrinsic dust reddening comes from what is already present in the five basis SEDs via the training set used to create the basis templates, and linear combinations thereof do not span the full range of realistic dust extinction observed in galaxy populations.

While these idealized conditions limit the realism of our mock data, they are irrelevant to the controlled experimental conditions of this study, if anything assuring that differentiation in the performance of the \pzpdf\ codes is due to the inferential techniques rather than nuances in the data.

\subsection{\lsst-like mock observations}
\label{sec:observations}

Given the SED, absolute $r$-band magnitude, and true redshift of each simulated galaxy, we computed apparent magnitudes in the six \lsst\ filter passbands, $ugrizy$.
We assigned magnitude errors in the six bands using the simple model of \citet{Ivezic:08}, assuming achievement of the full 10-year depth, with a modification of fiducial \lsst\ total numbers of 30-second visits for photometric error generation: we assume 60 visits in $u$-band, 80 visits in $g$-band, 180 visits in $r$-band, 180 visits in $i$-band, 160 visits in $z$-band, and 160 visits in $y$-band.

As a consequence of adding Gaussian-distributed photometric errors, 2.0 per cent of our galaxies exhibit a negative flux in one or more bands, the vast majority of which are in the $u$-band.
We deem such negative fluxes \textit{non-detections} and assign a placholder magnitude of 99.0 in the catalogue to indicate to the \pzpdf\ codes that such galaxies would be ``looked at but not seen'' in multi-band forced photometry.

The full dataset thus covers $400$ square degrees and contains $238$ million galaxies of redshift $0 < z \leq 8.7$ down to $r = 29$.
Systematic inconsistencies with galaxy colors at $z > 2$ were observed, so the catalogue was limited to $0 < z \leq 2.0$.
To obtain a catalogue matching the \lsst\ Gold Sample, we imposed a cut of $i < 25.3$, which gives a signal-to-noise ratio $\gtrsim 30$ for most galaxies.
In order for statistical errors to be subdominant to the systematic errors we aim to probe, we further reduced the sample size to $<10^{7}$ galaxies by isolating $\sim 16.8$ square degrees selected from five separate spatial regions of the simulation.
We refer to this final set of galaxies as DC1, for the first \lsstdesc\ Data Challenge.

\subsection{Shared prior information}
\label{sec:controlled}

For the purpose of performing a controlled experiment that compares \pzpdf\ codes on equal footing as a baseline for a future sensitivity analysis, we take care to provide each with optimal prior information.
Redshift estimation approaches built upon physical modeling and machine learning alike have a notion of prior information considered beyond the photometry of the data for which redshift is to be constrained: that information is derived from a template library for a model-based code and a training set for a data-driven code.
In this initial study, we seek to set a baseline for a later comparison of the performance of \pzpdf\ codes under incomplete and non-representative prior information that will propagate differently in the space of data-driven and model-based algorithms.
However, for the baseline case of perfect prior information, physical modeling and machine learning codes can indeed be put on truly equal footing.
We outline the equivalent ways of providing all codes perfect prior information below.

\subsubsection{Training and test set division}
\label{sec:buzztraining}

Following the findings of \citet{Bernstein:10}, \citet{Newman:2015}, and \citet{Masters:2015} that only $\sim\!10^{4}$ spectra are necessary to calibrate \pz s to Stage IV requirements, we aimed to set aside a randomly selected training set of $3-5\times 10^{4}$ galaxies, $\sim 10$ per cent of the full sample.
After all cuts described above, we designated the \textit{DC1 training set} of $44\,404$ galaxies for which observed photometry, true SEDs, and true redshifts would be provided to all codes and the blinded \textit{DC1 test set} of $399\,356$ galaxies for which photometry alone would be provided to all codes and \pzpdf s would be requested.
The exact form of \lsst\ photometric filter transmission curves were also considered public information that could be used by any code.

\subsubsection{Template library construction}
\label{sec:buzztemplates}

We aimed to provide template-fitting codes with complete yet manageable library of templates spanning the space of SEDs of the DC1 galaxies.
We constructed $K=100$ representative templates from the $\sim 5 \times 10^{5}$ SEDs of the \sdss\ DR6 NYU-VAGC by using the five-dimensional vectors of SED weight coefficients described above.
After regularizing the SED weight coefficients $\in [0, 1]$, we ran a simple K-means clustering algorithm on the five-dimensional space of regularized SED weight coefficients of the \sdss\ galaxy sample.
The resulting clusters were used to define Voronoi cells in the space of weight coefficients, with centre positions corresponding to weights for the \texttt{k-correct} SED components, yielding the 100 SEDs that comprise the \textit{DC1 template set} provided to all template-based codes.
We did not, however, exclude from consideration template-based codes that made modifications in their use of these templates due to architecture limitations (as opposed to knowledge of the experimental conditions that could artificially boost the code's apparent performance), with deviations noted in Section~\ref{sec:pzcodes}.
