\section{The simulation and mock galaxy catalog}
\label{sec:sims}
%(Sam Schmidt, Eve Kovacs, Tina Peters)

In order to test the current generation codes, we employ a simulated galaxy catalogue. The simulation is completely catalogue-based, with no image construction or mock measurements made. We describe these in detail below.

\subsection{Buzzard-v1.0 simulation}
\label{sec:buzzard}
The \textsc{Buzzard-highres-v1.0} \red{put in cites to in prep Buzzard papers} catalogue construction started with a dark matter only simulation. This N-body simulation contained $2048^3$ particles in a $400$ Mpc h$^{-1}$ box. \red{[N]} snapshots (with smoothing and interpolation between snapshots) were saved in order to construct a lightcone. Dark matter halos were identified using the \textsc{Rockstar} software package \citep{Behroozi:13}. These dark matter halos were populated with galaxies with a stellar mass and absolute $r$-band magnitude in the SDSS system determined using a sub-halo abundance matching model constrained to match both projected two-point galaxy clustering statistics and an observed conditional stellar mass function \citep{Reddick:13}.

To assign an SED to each galaxy, the {\it Adding Density Dependent Spectral Energy Distributions} (\textsc{ADDSEDS}, deRose in prep.)\footnote{\url{https://github.com/vipasu/addseds}} procedure was used. This consisted of training an empirical relation between absolute $r$-band magnitude, local galaxy density, and SED using a sample of $\sim 5e^{5}$ galaxies from the magnitude-limited Sloan Digital Sky Survey Data Release 6 Value Added Galaxy Catalog \citep{Blanton:05}\red{[Note: is this the proper reference to SDSS-NYU VAGC? File is called combined\_dr6\_cooper.fits, but I don't see which Cooper et al 2006 this is supposed to refer to?]}. Each SDSS spectrum is fit with a sum of five SED components using the \textsc{k-correct \red{v?}} software package\footnote{\url{http://kcorrect.org}} \citep{Blanton:07}, thus each galaxy SED is parameterized as five weights for the basis SEDs. The distance to the spatial projected fifth-nearest neighbour was used as a proxy for local density in the SDSS training sample. For each simulated galaxy, a ``random'' \red{[details]} galaxy with ``similar'' \red{[details]} absolute $r$-band magnitude and local galaxy density was chosen from the training set, and that training galaxy's SED was assigned to the simulated galaxy. Given the SED, absolute $r$-band magnitude and redshift, we computed apparent magnitudes in the six LSST filter passbands, $ugrizy$. We assigned magnitude errors in the six bands using the simple model described in \citet{Ivezic:08}, assuming full 10-year depth observations had been completed.  The number of total 30-second visits assumed when generating the photometric errors differs slightly from the fiducial numbers assumed for LSST: we assume 60 visits in u-band, 80 visits in g-band, 180 visits in r-band, 180 visits in i-band, 160 visits in z-band, and 160 visits in y-band.
%numbers taken from Alex Abate's photErrorModel.py script in PhotoZDC1 repository, which was used with nYrObs=10.

\subsubsection{Selection of training and test sets}
\label{sec:buzztraining}
The total catalogue covered $400$ square degrees and contained $238$ million galaxies to an apparent magnitude limit of $r\!=\!29$ and spanning the redshift range $0\!<\!z\!\leq\!8.7$. This catalogue contained two orders of magnitude more galaxies than were needed for this study, so only $\sim\!8$ square degrees were used. Systematic problems with galaxy colors above $z\!>\!2$ were observed, so the catalogue was trimmed to include only galaxies in the redshift range $0\!<\!z\!\leq\!2.0$. A random subset of the the remaining galaxies was chosen, and placed at random into either a ``training'' set ($10$ per cent of the sample), for which the galaxies true redshifts will be supplied, or a ``test'' set (the remaining $90$ per cent of the sample), for which each code will need to predict a redshift PDF for each galaxy. The resulting catalogues contain $111\,171$ training galaxies and $1\,000\,883$ test galaxies.  We restrict our analysis to a sample with $i<25.3$, which give a signal-to-noise $\sim\,30$ for most galaxies, a cut often referred to as the expected ``LSST Gold Sample''.  This magnitude cut results in a training set with $44\,404$ galaxies and a test set containing $399\,356$ galaxies.  All subsequent results will evaluate this ``gold sample'' test set.

\subsubsection{Templates}
\label{sec:buzztemplates}
As mentioned in Section~\ref{sec:buzzard}, the SEDs in the Buzzard simulation are drawn from an empirical set of SEDs taken from the SDSS DR6 NYU-VAGC, a sample of roughly $\sim5e^{5}$ galaxies with spectra in SDSS. To determine a finite set of templates to use with template fitting codes we take the five SED weight coefficients for each of the $\sim 500\,000$ galaxies in the SDSS sample and run a simple K-means clustering algorithm on this five dimensional space. The K-means cluster centres span the space of coefficients and properly reflect the underlying density in the coefficient space, thus providing a reasonable approximation for a spanning SED set. An ad-hoc number of $K=100$ was chosen and the $100$ K-means centre positions are taken as the weights for the \textsc{k-correct} SED components to construct one hundred template SEDs. These $100$ templates were provided,
%\red{(JS: if I remembered correctly, weren't there 150 templates given? Answer: no, initially I supplied a list of 150 that were drawn with a different algorithm, but once we switched to the final k-means set I only supplied 100 (the initial algorithm over-weighted outlier SEDs, which is why we needed the extra 50)--SJS)}
however not every template code uses this set of one hundred templates: because EAZY was designed and written to use the same five basis templates employed by \textsc{k-correct} when constructing our mock galaxies, EAZY was run using linear combinations of these five templates rather than using the 100 discrete templates.


\subsubsection{Limitations}
\label{sec:buzzlimitations}
For our initial investigation of photometric redshift codes, we begin with a data set that is somewhat idealized, and does not contain all of the complicating factors present in real data.  In several cases, the simplification is done with a purpose, with potentially confounding effects excluded in order to better isolate the differences between current-generation photo-$z$ codes, and their causes.  We list several of the simulations limitations in this section.
As the simulation is catalogue-based, no image level effects, such as photometric measurement effects, object blending, contamination from sky background (Zodiacal light, scattered light, etc...), lensing magnification, or Galactic reddening are included.  No stars are included in the catalogue, nor are the effects of AGN.
As all SEDs are constructed from only five basis templates, properties of the galaxy population will be restricted to follow linear combinations of the characteristics of the five basis templates, so certain non-linear features, for example the full range of emission line fluxes relative to the continuum, will not be included in the model galaxy population.  No additional dust reddening intrinsic to the host galaxy is included, the only approximation of dust extinction comes in the form of dust encoded in the five basis SEDs via the training set used to create the basis templates.  Simple linear combinations of these basis templats will, once again, not explore the full range of realistic dust extinction observed in galaxy populations.

%\subsection{ABC-Galacticus simulation}
%\label{sec:galacticus}
%\red{The Argonne-Berkeley-Carnegie (ABC) Galacticus cosmological simulation uses a semi-analytic model to simulate the galaxy properties and is not directly tuned to observations. Each simulated SED is replaced with a SED drawn from the distribution of empirical SEDs given by the SED Atlas of \citet{Brown:14} (diffusion maps method in development).}
