\section{Data}
\label{sec:sims}

In order to test the current generation of \pzpdf\ codes, we employ an existing simulated galaxy catalogue, described in detail in Section~\ref{sec:buzzard}.
The experimental conditions shared among all codes are motivated by the \lsst\ SRD requirements and implemented for machine learning and template-based \pzpdf\ codes according to the procedures of Sections~\ref{sec:buzztraining} and \ref{sec:buzztemplates} respectively.
Finally, in Section~\ref{sec:limitations}, we elaborate on the scope of the experimental conditions considered as a result of the sophistication of the mock data.

\subsection{Buzzard-v1.0 simulation}
\label{sec:buzzard}

Our mock catalogue is derived from the \textsc{Buzzard}-highres-v1.0  of De Rose et al., in prep, Wechsler et al., in prep) catalogue.
% UPDATE THIS CITATION NOW THAT THE PAPER IS OUT
\textsc{Buzzard} is built on a dark matter-only N-body simulation of $2048^{3}$ particles in a $400$ Mpc h$^{-1}$ box.
The lightcone was constructed from smoothing and interpolation between a set of time snapshots.
Dark matter halos were identified using the \texttt{Rockstar} software package \citep{Behroozi:13} and then populated with galaxies with a stellar mass and absolute $r$-band magnitude in the \sdss\ system determined using a sub-halo abundance matching model constrained to match both projected two-point galaxy clustering statistics and an observed conditional stellar mass function \citep{Reddick:13}.

To assign an SED to each galaxy, the Adding Density Dependent Spectral Energy Distributions (\texttt{ADDSEDS}, deRose in prep.)\footnote{\url{https://github.com/vipasu/addseds}} procedure was used.
\texttt{ADDSEDS} uses a sample of $\sim 5\times 10^{5}$ galaxies from the magnitude-limited \sdss\ Data Release 6 Value Added Galaxy Catalogue \citep{Blanton:05} to train an empirical relation between absolute $r$-band magnitude, local galaxy density, and SED.
Each \sdss\ spectrum is parameterized by five weights corresponding to a weighted sum of five basis SED components using the \texttt{k-correct \red{v4\_3?}} software package\footnote{\url{http://kcorrect.org}} \citep{Blanton:07}.

Correlations between SED and galaxy environment were included so as to preserve the colour-density relation of galaxy environment.
The distance to the spatially projected fifth-nearest neighbour was used as a proxy for local density in the \sdss\ training sample.
For each simulated galaxy, a galaxy with similar absolute $r$-band magnitude and local galaxy density was chosen from the training set, and that training galaxy's SED was assigned to the simulated galaxy.

Given the SED, absolute $r$-band magnitude, and redshift, we computed apparent magnitudes in the six \lsst\ filter passbands, $ugrizy$.
We assigned magnitude errors in the six bands using the simple model of \citet{Ivezic:08}, assuming achievement of the full 10-year depth, with a modification of fiducial \lsst\ total numbers of 30-second visits for photometric error generation: we assume 60 visits in $u$-band, 80 visits in $g$-band, 180 visits in $r$-band, 180 visits in $i$-band, 160 visits in $z$-band, and 160 visits in $y$-band.
%numbers taken from Alex Abate's photErrorModel.py script in PhotoZDC1 repository, which was used with nYrObs=10.

As a consequence of adding Gaussian-distributed photometric errors, 2.0\% of our galaxies exhibit a negative flux in one or more bands, the vast majority of which are in the $u$-band.
We deem such negative fluxes \textit{non-detections} and assign a placholder magnitude of 99.0 in the catalogue to indicate to the \pzpdf\ codes that such galaxies would be ``looked at but not seen'' in multi-band forced photometry.

The full \textsc{Buzzard} dataset thus covers $400$ square degrees and contains $238$ million galaxies of redshift $0 < z \leq 8.7$ down to $r = 29$.
Systematic inconsistencies with galaxy colors at $z > 2$ were observed, so the catalogue was limited to $0 < z \leq 2.0$.
To obtain a catalogue matching the \lsst\ Gold Sample, we imposed an cut of $i < 25.3$, which gives a signal-to-noise ratio $\sim 30$ for most galaxies.
In order for statistical errors to be subdominant to the systematic errors we aim to probe, we further reduced the sample size to $<10^{7}$ galaxies by isolating $\sim 16.8$ square degrees selected from five separate spatial regions of the simulation.

\subsection{Experimental Conditions}
\label{sec:controlled}

% NOW MOTIVATE THE SHARED PRIOR INFORMATION
For the purpose of performing a controlled experiment that compares \pzpdf\ codes on equal footing as a baseline for a future sensitivity analysis, we take care to provide each with maximally optimistic prior information.
Redshift estimation approaches built upon physical modeling and machine learning alike have a notion of prior information considered beyond the photometry of the data for which redshift is to be constrained: that information is derived from a template library for a model-based code and a training set for a data-driven code.
In this initial study, we seek to set a baseline for a later comparison of the performance of \pzpdf\ codes under incomplete and non-representative prior information that will propagate differently in the space of data-driven and model-based algorithms.
However, for the baseline case of perfect prior information, physical modeling and machine learning codes can indeed be put on truly equal footing.
We outline the equivalent ways of providing all codes perfect prior information below.

\subsubsection{Training and test set division}
\label{sec:buzztraining}

Following the findings of \citet{Bernstein:10}, \citet{Masters:2017} that only $~10^{4}$ spectra are necessary to calibrate \pz s to Stage IV requirements, we aimed to set aside a randomly selected training set of $3-5\times 10^{4}$ galaxies, $\sim 10\%$ of the full sample.
After all cuts described above, we designated a training set of $44\,404$ galaxies for which both photometry and redshifts would be provided to data-driven codes and a blinded test set of $399\,356$ galaxies for which photometry alone would be provided to all codes and \pzpdf s would be requested.
%The resulting catalogues contain $111\,171$ training galaxies and $1\,000\,883$ test galaxies.

\subsubsection{Template library construction}
\label{sec:buzztemplates}

We aimed to provide template-fitting codes with complete yet manageable library of templates spanning the space of SEDs of the \textsc{Buzzard} galaxies.
We constructed $K=100$ representative templates from the $\sim 5 \times 10^{5}$ SEDs of the \sdss\ DR6 NYU-VAGC by using the five-dimensional vectors of SED weight coefficients described above.
After regularizing the SED weight coefficients $\in [0, 1]$, we ran a simple K-means clustering algorithm on the five-dimensional space of regularized SED weight coefficients of the \sdss\ galaxy sample.
The resulting clusters were used to define Voronoi cells in the space of weight coefficients, with centre positions corresponding to weights for the \texttt{k-correct} SED components, yielding 100 templates to be provided to all template-based codes.
We did not, however, exclude from consideration template-based codes that made modifications in their use of these templates due to architecture limitations (as opposed to knowledge of the experimental conditions that could artificially boost the code's apparent performance), with deviations noted in Section~\ref{sec:pzcodes}.

\subsection{Limitations}
\label{sec:buzzlimitations}

For our initial investigation of photometric redshift codes, we begin with a data set that is somewhat idealized, and does not contain all of the complicating factors present in real data.
In several cases, the simplification is done with a purpose, with potentially confounding effects excluded in order to better isolate the differences between current-generation photo-$z$ codes and their causes.
We list several of the simulation limitations in this section.
As the simulation is catalogue-based, no image level effects, such as photometric measurement effects, object blending, contamination from sky background (Zodiacal light, scattered light, etc...), lensing magnification, or Galactic reddening are included.  No stars are included in the catalogue, nor are the effects of AGN.
As all SEDs are constructed from only five basis templates, properties of the galaxy population will be restricted to follow linear combinations of the characteristics of the five basis templates, so certain non-linear features, for example the full range of emission line fluxes relative to the continuum, will not be included in the model galaxy population.
Moreover, the linear combinations of templates are modeled on the $\sim5\times 10^{5}$ SDSS galaxies discussed in Section~\ref{sec:buzzard}, and thus only galaxies that resemble those spectroscopically observed by the SDSS will be included in the sample.
No additional dust reddening intrinsic to the host galaxy is included, the only approximation of dust extinction comes in the form of dust encoded in the five basis SEDs via the training set used to create the basis templates.
Simple linear combinations of these basis templates will, once again, not explore the full range of realistic dust extinction observed in galaxy populations.
While these idealized conditions limit the realism of our galaxy population, some are also by design.
We aim to test the photo-z codes at a very basic level, and a simplified model assures that differences in results seen between the codes are due to fundamental differences in their underlying assumptions and implementation details, rather than more nuanced properties.
