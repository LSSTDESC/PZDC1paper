\section{Discussion and future work}
\label{sec:discussion}
In contrast with other \pzpdf\ comparison papers that have aimed to identify the ``best'' code for a given survey, we have focused on the somewhat more philosophical questions of how to assess \pzpdf\ methods and how to interpret differences between codes in terms of \pzpdf\ performance.
In Section~\ref{sec:caution}, we reframe the strong performance of our pathological \pzpdf\ technique, \trainz, as a cautionary tale about the importance of choosing appropriate comparison metrics.
In Section~\ref{sec:futureexperiments}, we outline the experiments we intend to build upon this study.
In Section~\ref{sec:futuredata}, we discuss the enhancements of the mock data set that will be necessary to enable the future experiments.


\subsection{Interpretation of metrics}
\label{sec:caution}

We remind the reader that codes utilized in this study were given a goal of obtaining accurate \pzpdf s, not an accurate stacked estimator of the redshift distribution, so we do not expect the same codes to necessarily perform well for both classes of metrics.
Indeed, the codes were optimized for their interpretation of our request for ``accurate \pzpdf s,'' and we expect that the implementations would have been adjusted had we requested optimization of the traditional metrics of Appendices~\ref{sec:moments} and~\ref{sec:pointmetrics}.

Furthermore, our metrics are not necessarily able to assess the fidelity of individual \pzpdf s relative to true posteriors: in the absence of a ``true PDF'' from which redshifts are drawn, it is difficult to construct metrics to measure performance for individual galaxies rather than ensembles.
(The CDE Loss metric of section~\ref{sec:CDE_loss} is an exception to this rule.)
A lack of appropriate metrics more sophisticated than the CDE Loss remains an open issue for science cases requiring accurate individual galaxy PDFs.
The metric-specific performance demonstrated in this paper implies that we may need multiple \pzpdf\ approaches tuned to each metric in order to maximize returns over all science cases in large upcoming surveys.

The \trainz\ estimator of Section~\ref{sec:trainz}, which assigns every galaxy a \pzpdf\ equal to $N(z)$ of the training set, is introduced as an experimental control or null test to demonstrate this point via \textit{reductio ad absurdum}.
Because our training set is perfectly representative of the test set, $N(z)$ should be identical for both sets down to statistical noise.
\textbf{We make the alarming observation that \trainz, the absurd experimental control, outperforms all codes on the CDF-based metrics, and all but one code on the $N(z)$ based statistics.}
The PIT and other CDF-based metrics upon which modern \pzpdf\ comparisons are built \citep{Bordoloi:10,Polsterer:16,Tanaka:17} can be gamed by a trivial estimator that yields only an affirmation of prior knowledge uninformed by the data.
\boldblue{The extreme case of \trainz\ also exemplifies the fact that characteristics of the training data are imprinted on the ensemble distributions via the prior, and any imperfections and biases in the training sample will be propagated to the measured \pzpdf s.}
In other words, such ensemble metrics are insufficient for the task of selecting \pzpdf\ codes for analysis pipelines\footnote{That being said, we note that close relatives of \trainz\ have been employed by weak lensing surveys of the past and present \citep{Lima:08, Hildebrandt:17, Hoyle:18} to estimate $N(z)$ by assigning each test set galaxy the redshift distribution of a \emph{subset} of the training set, where the subset is defined as similar to the test set galaxy in the space of photometric data.
The specific science goal naturally guides the choice of metric to focus on $N(z)$ rather than individual \pzpdf s, and on the basis of that metric such improvements upon \trainz\ are guaranteed to be more robust to training set imbalance.}.

The CDE loss and point estimate metrics appropriately penalize \trainz's naivete.
As shown in Appendix~\ref{sec:pointmetrics}, \trainz ~has identical $ZPEAK$ and $ZWEIGHT$ values for every galaxy, and thus the \pz\ point estimates are constant as a function of true redshift, i.e.~a horizontal line at the mode and mean of the training set distribution respectively.
The explicit dependence on the individual posteriors in the calculation of the CDE loss, described in Section~\ref{sec:cdelossresults}, distinguishes this metric from those of the \pzpdf\ ensemble and stacked estimator of the redshift distribution, despite their prevalence in the \pz\ literature.

In summary, context is crucial to defend against deceptively strong performers such as \trainz;
the best \pzpdf\ method is the one that most effectively achieves the science goals of a particular study, not the one that performs best on a metric that does not reflect those goals.
In the absence of a single scientific motivation or the information necessary for a principled metric definition, we must consider many metrics and be critical of the information transmitted by each.

\subsection{Extensions to the experimental design}
\label{sec:futureexperiments}

The work presented in this paper is only a first step in assessing \pzpdf\ approaches and moving toward a photometric redshift estimator that will be employed for LSST analyses.  Extensions of the experimental design will require further rounds of analyses, and the authors welcome interest from those outside \lsstdesc\ to have their codes assessed in these future investigations.

This initial paper explores \pzpdf\ code performance in idealized conditions with perfect catalogue-based photometry and representative training data, but the resilience of each code to realistic imperfections in prior information has not yet been evaluated.
A top priority for a follow-up study is to test realistic forms of incomplete, erroneous, and non-representative template libraries and training sets as well as the impact of other forms of external priors that must be ingested by the codes, major concerns in \citet{Newman:2015, Masters:2017}.
Outright redshift failures due to emission line misidentification or noise spikes may be modeled by the inclusion of a small number of high-confidence yet false redshifts.
We plan to perform a full sensitivity analysis on a realistically incomplete training set of spectroscopic galaxies, modeling the performance of spectrographs, emission-line properties, and expected signal-to-noise to determine which potential training set galaxies are most likely not to yield a secure redshift.

Appendix~\ref{sec:moments} only addresses the stacked estimator of the redshift distribution of the entire galaxy catalogue rather than subsets in bins, tomographic or otherwise.
The effects of tomographic binning schemes will be explored in a dedicated future paper, including propagation of redshift uncertainties in a set of fiducial tomographic redshift bins in order to estimate impact on cosmological parameter estimation.

Sequels to this study will also address some shortcomings of our experimental procedure.
The fixed redshift grid shared between the codes may have unfairly penalized codes with a different native parameterization, as precision is lost when converting between formats.
Performance on the (admittedly small) population of sharply peaked \pzpdf s may have been suppressed across all codes due to the insufficient resolution of the redshift grid.
In light of the results of \citet[]{Malz:2018}, in future analyses we plan to switch from a fixed grid to the quantile parameterization or to permit each code to use its native storage format under a shared number of parameters.

Section~\ref{sec:metrics} discussed the difficulty in evaluating PDF accuracy for individual objects with known $(z, d)$ information but without a known $p(z, d)$.
In a follow-up study, we will
generate mock data probabilistically, yielding true PDFs in addition to true redshifts and photometric data.
This future data set will enable tests of PDF accuracy for individual galaxies rather than solely ensembles.

\subsection{Realistic mock data}
\label{sec:futuredata}

To make optimal use of the \lsst\ data for cosmological and other astrophysical analyses of the \lsstdesc\ SRM, future investigations that build upon this one will require a more sophisticated set of galaxy photometry and redshifts.
This initial paper explored a data set that was constructed at the catalogue level, with no inclusion of the complications that arise from photometric measurements of imaging data.
Future data challenges will move to catalogues constructed from mock images, including the complications of deblending, sensor inefficiencies, and heterogeneous observing conditions, all anticipated to affect the measured colours of \lsst's galaxy sample \citep{Dawson:2016}.

The DC1 galaxy SEDs were linear combinations of just five basis SED templates, and the next generation of data for \pzpdf\ investigations must include a broader range of physical properties.
Though we only considered $z < 2$ here, \lsst\ 10-year data will contain $z > 2$ galaxies, plagued by fainter apparent magnitudes and anomalous colours due to stellar evolution.
A subsequent study must also have a data set that includes low-level active galactic nuclei (AGN) features in the SEDs, which perturb colours and other host galaxy properties.
An observational degeneracy between the Lyman break of a $z \sim 2-3$ galaxy from the Balmer break of a $z \sim 0.2-0.3$ galaxy is a known source of catastrophic outliers \citep{Massarotti:2001} that was not effectively included in this study.
To gauge the sensitivity of \pzpdf\ estimators to catastrophic outliers, our data set must include realistic high-redshift galaxy populations.

