\section{Conclusion}
\label{sec:conclusion}

This paper compares twelve \pzpdf\ codes under controlled experimental conditions of representative and complete prior information to set a baseline for an upcoming sensitivity analysis.
This work isolates the impact on metrics of \pzpdf\ accuracy due to the estimation technique as opposed to the complications of realistic physical systematics of the photometry.
Though the mock data set of this investigation did not include true \pz\ posteriors for comparison, \textbf{we interpret deviations from perfect results given perfect prior information as the imprint of the implicit assumptions underlying the estimation approach}.

We evaluate the twelve codes under science-agnostic metrics both established and emerging to stress-test the ensemble properties of \pzpdf\ catalogues derived by each method.
In appendices, we also present metrics of point estimates and a prevalent summary statistic of \pzpdf\ catalogues used in cosmological analyses to enable the reader to relate this work to studies of similar scope.
We observe that no one code dominates in all metrics, and that the standard metrics of \pzpdf s and the stacked estimator of the redshift distribution can be gamed by a very simplistic procedure that asserts the prior over the data.
We emphasize to the \pz\ community that \textbf{metrics used to vet \pzpdf\ methods must be scrutinized to ensure they correspond to the quantities that matter to our science}.
