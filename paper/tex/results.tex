\section{Results}
\label{sec:results}
%(Sam Schmidt, Bryce Kalmbach, Johann Cohen Tanugi, Rongpu Zhou, Alex Malz)

We begin with a demonstrative visual inspection of the \pzpdf s produced by each code for individual galaxies.
Figure~\ref{fig:pz_examples} shows the \pzpdf s for four galaxies chosen as examples of \pzpdf\ archetypes: a narrow unimodal PDF, a broad unimodal PDF, a bimodal PDF, and a multimodal PDF.
We reiterate that under our idealized experimental conditions, differences between codes are the isolated signature of the implicit prior due to the method by which the \pzpdf s were derived.

\begin{figure*}
%\centering
\includegraphics[width=0.49\textwidth]{fig/pz_12codes_261931_noseaborn_crop.jpg}\includegraphics[width=0.49\textwidth]{fig/pz_12codes_471167_noseaborn_crop.jpg}\\
\includegraphics[width=0.49\textwidth]{fig/pz_12codes_713178_noseaborn_crop.jpg}\includegraphics[width=0.49\textwidth]{fig/pz_12codes_982747_noseaborn_crop.jpg}
\caption{The individual \pzpdf s (blue) distributions produced by the twelve codes (small panels) on four exemplary galaxies' photometry (large panels) with different true redshifts (red).
The \pzpdf s of all codes share some features for the example galaxies due to physical color degeneracies and photometric errors: tight unimodal $p(z)$ (upper left), broad unimodal $p(z)$ (upper right), bimodal $p(z)$ (lower right), and complex/multimodal $p(z)$ (lower left).
The diverse algorithms and implementations induce differences in small-scale structure and sensitivity to physical systematics.}
\label{fig:pz_examples}
\end{figure*}

The most striking differences between codes are due to small-scale features induced by the interaction between the shared piecewise constant parameterization of $200$ bins $0 < z < 2$ of Section~\ref{sec:metrics} and the smoothing conditions or lack thereof in each algorithm.
The $\rm{d}z = 0.01$ redshift resolution is sufficient to capture the broad peaks of faint galaxies' \pzpdf s with large photometric errors but is too broad to resolve the narrow peaks for bright galaxies' \pzpdf s with small photometric errors.
This observation is consistent with the findings of \citet[]{Malz:qp} that the piecewise constant parameterization underperforms in the presence of small-scale structures.

However, the shared small-scale features of \annz, \metaphor, \cmnn, and \skynet\ are a result of various weighted sums of the limited number of training set galaxies with colors similar to those of the test set galaxy in question, with behavior closer to classification than regression in the case of \annz.
The settings used on \gpz\ in this work forced broadening of the single Gaussian to cover the multimodal redshift solutions of the other codes.
%Interestingly, while \textsc{ANNz2} shows an abundance of small scale structure in individual $p(z)$ measurements (see Fig.~\ref{fig:pz_examples}), the summed $\hat{N}(z)$ is rather smooth, where the small scale features average out.  This is not the case for the two other codes that show an abundance of substructure in their individual $p(z)$: both \textsc{CMNN} and \textsc{SkyNet} show small scale features both in $p(z)$ and $\hat{N}(z)$.
%In contrast, FlexZBoost, for example, can return estimates on any grid without compression errors as it’s a basis expansion method where only the expansion coefficients need to be stored.
%Codes with a native output format other than the shared piecewise constant binning scheme (or one that can be losslessly converted to it) may suffer from loss of information when converting to it, which could artificially favor some codes over others in a limited number of cases, for example bright galaxies with very narrow $p(z)$ where the true peak falls between grid points.  We will discuss PDF storage in Section~\ref{sec:discussion}.
%Furthermore, the fidelity of photo-$z$ interim posteriors in this format varies with the quality of the photometry.
%Switching to a quantile based parameterization may be more costly computationally, for example template-based codes would need to test more grid point in order to resolve the quantiles for bright galaxies.  However, the computational time for template based codes scales roughly linearly with the number of grid points, so this may be a reasonable thing to implement.
% We will discuss this further in Section~\ref{sec:discussion}.
% \red{someone review this statement to make sure that I'm saying this correctly!}

\subsection{Performance on \pzpdf\ ensembles}
\label{sec:pitqq}

Figure~\ref{fig:pitqq} shows a histogram of PIT values, QQ plot, and QQ difference plot relative to the ideal diagonal, showcasing the biases and trends in the average accuracy of the \pzpdf s for each code.
The high QQ values (more high than low PIT values) of \bpz, \cmnn, \delight, \eazy, and \gpz\ indicate \pzpdf s biased low, and the low QQ values (more low than high PIT values) of \skynet\ and \tpz\ indicate \pzpdf s biased high.

\begin{figure*}
\centering
\includegraphics[width=0.74\textwidth]{fig/PITANDQQplot_12codes_crop.jpg}
\caption{The QQ plot (red) and PIT histogram (blue) of the \pzpdf\ codes (panels) along with the ideal QQ (black dashed diagonal) and ideal PIT (gray horizontal) curves, as well as a difference plot for the QQ difference from the ideal diagonal (lower inset).
The twelve codes exhibit varying degrees of four deviations from perfection: an overabundance of PIT values at the centre of the distribution indicate a catalogue of overly broad \pzpdf s, an excess of PIT values at the extrema indicates a catalogue of overly narrow \pzpdf s, catastrophic outliers manifest as overabundances at PIT values of 0 and 1, and asymmetry indicates systematic bias, a form of model misspecification.}
\label{fig:pitqq}
\end{figure*}

The PIT histograms of \delight, \cmnn, \skynet, and \tpz\ feature an underrepresentation of extreme values, indicative of overly broad \pzpdf s, while the overrepresentation of extreme values for for \metaphor indicate overly narrow \pzpdf s.
These five codes in particular have a free parameter for bandwidth, which may be responsible for this vulnerability, in spite of the opportunity for fine-tuning with perfect prior information.
\flexzboost's ``sharpening'' parameter (described in Section~\ref{sec:flexzboost}) played a key role in diagonalizing the QQ plot, indicating a common avenue for improvement in the approaches that share this type of parameter.
On the other hand, the three purely template-based codes, \bpz, \eazy, and \lephare, do not exhibit much systematic broadening or narrowing, which may indicate that complete template coverage effectively defends from these effects.

\begin{table}
\setlength{\tabcolsep}{2pt}
\centering
\caption{The catastrophic outlier rate as defined by extreme PIT values.
We expect a value of 0.0002 for a proper Uniform distribution.
An excess over this small value indicates true redshifts that fall outside the non-zero support of the $p(z)$.}
\label{tab:pitoutlier}
\begin{tabular}{lc}
\hline
\hline
\Pz\ Code & fraction PIT$<10^{-4}$ or $>$0.9999\\
\hline
\annz       & 0.0265\\
\bpz        & 0.0192\\
\delight    & 0.0006\\
\eazy       & 0.0154\\
\flexzboost & 0.0202\\
\gpz        & 0.0058\\
\lephare    & 0.0486\\
\metaphor   & 0.0229\\
\cmnn       & 0.0034\\
\skynet     & 0.0001\\
\tpz        & 0.0130\\
\hline
\trainz     & 0.0002\\
\end{tabular}
\end{table}

Though the spikes in the first and last bin of the PIT histogram were cut off in Figure~\ref{fig:pitqq} for visualization, the catastrophic outlier rates are provided in Table~\ref{tab:pitoutlier}.
As expected, \trainz\ achieves precisely the 0.0002 value expected of an ideal PIT distribution.
\annz, \flexzboost, \lephare, and \metaphor\ have notably high catastrophic outlier rates $> 0.02$, exceeding 100 times the ideal PIT rate, meriting further investigation.

\begin{figure*}
\centering
\includegraphics[width=0.74\textwidth]{fig/KSvsCvMvsAD_PIT_withnull_jpg.jpg}
\caption{A visualization of the Kolmogorov-Smirnoff (KS, blue diamond), Cramer-von Mises (CvM, black star), and Anderson-Darling (AD, red asterisk) statistics for the PIT distributions.
There is generally good agreement between these statistics, with differences corresponding to the codes with outstanding catastrophic outlier rates, a reflection in the differences in how each statistic weights the tails of the distribution.}
\label{fig:pit_stats}
\end{figure*}

Figure~\ref{fig:pit_stats} displays the values of the KS, CvM, and AD test statistics between the PIT distribution and a uniform distribution $U(0, 1)$, highlighting the relative rather than absolute numbers.
%\red{Can p-values be supplied for each statistic? The statistics themselves are difficult to interpret, other than ``lower is better'' (p-value in skgof was broken, having trouble finding 1-sample KS calculation for uniform distribution)}
\metaphor\ and \lephare\ perform well under the AD but poorly under the KS and CvM due to their high catastrophic outlier rates.
\annz\ and \flexzboost\  are the top scorers under these metrics of the PIT distribution.
\annz's strong performance can be attributed to an aspect of the training process in which training set galaxies with a PIT that more closely matches the percentiles of the DC1 training set's redshift distribution are upweighted; in effect, these quantile-based metrics were part of the algorithm itself that may or may not serve it well under more realistic experimental conditions.

\subsection{Performance on individual \pzpdf s}
\label{sec:cdelossresults}

The values of the CDE loss statistic of individual \pzpdf\ accuracy are provided in Table~\ref{tab:cdeloss}.
It is worth noting that strong performance on the CDE loss should imply strong performance on the other metrics, though the inverse is not necessarily true.
Thus the CDE loss is the most effective metric for generic science cases.

\begin{table}  %%% DATA TABLE %%%
\centering
\caption{CDE loss statistic of the individual \pzpdf s for each code.
A lower value of the CDE loss indicates more accurate individual \pzpdf s, with \cmnn\ and \flexzboost\ performing best under this metric.}
\label{tab:cdeloss}
\begin{tabular}{lr}
\hline
Photo-$z$ Code & CDE Loss \\
\hline
\annz 	    & $-6.88$ \\
\bpz 		    & $-7.82$ \\
%\textsc{Delight} 	& $-4.06$ \\
\delight    & $-8.33$\\
%\textsc{EAZY} 		& $-7.97$ \\
\eazy       & $-7.07$ \\
%\textsc{FlexZBoost} & $-11.51$ \\
\flexzboost & $-10.60$\\
\gpz		    & $-9.93$ \\
\lephare 	  & $-1.66$ \\
\metaphor 	& $-6.28$ \\
%\textsc{CMNN} 		& $-$ \\
\cmnn       & $-10.43$ \\
\skynet 	  & $-7.89$ \\
\tpz 		    & $-9.55$ \\
\hline
\trainz		  & $-0.83$ \\
%\textsc{Frankenz}	& $-$  \\
%\hline
\end{tabular}
\end{table}

This metric is the only one that can appropriately penalize \trainz\ and indicates strong performance for \cmnn\ and \flexzboost, the latter of which is optimized for this metric.
%Empirically, we have found that PIT RMSE is not as closely correlated to CDE loss as it is to the $N(z)$ statistics.

%\subsection{Response of Individual Codes}
%\label{sec:res:pz_indiv_codes}
%\red{this may be incorporated in to 5.1 and 5.2 rather than live in its own section!}


%\red{HOW CODES deal with negative fluxes, and magnitude uncertainties}

%\red{overall conclusions of the Results section}
